\documentclass[11pt]{amsart}

\usepackage{macros}

\tikzset{%
    symbol/.style={%
        ,draw=none
        ,every to/.append style={%
            ,edge node={%
                node [%
                    ,sloped
                    ,allow upside down
                    ,auto=false
                    ]{$#1$}
                }
            }
        }
    }

\linespread{1.25}

\usepackage[final]{pdfpages}

\setcounter{tocdepth}{2}

\title{Lecture 7: Model categories, II}

\def\Fun{{\sf Fun}}
\def\colim{{\rm colim}}
\def\Set{{\sf Set}}
\def\Top{{\sf Top}}
\def\Cyl{{\rm Cyl}}
\def\ho{{\rm ho}}
\def\Ran{{\rm Ran}}
\def\Lan{{\rm Lan}}

\begin{document}
\maketitle

We begin by covering some more basic categorical facts relating to model categories. 
By definition, a {\em cobase change} of a map $f : X \to Y$ in any category, is a map $g : W \to Z$ that fits in pushout diagram
\[
\begin{tikzcd}
X \ar[d,"f"] \ar[r] & W \ar[d,"g"] \\
Y \ar[r] & Z .
\end{tikzcd}
\]
Similarly, a {\em base change} of a map $f : X \to Y$ is a map $g : W \to Z$ that sits in a pullback diagram
\[
\begin{tikzcd}
W \ar[d,"g"] \ar[r] & X \ar[d,"f"] \\
Z \ar[r] & Y .
\end{tikzcd}
\]

\begin{prop}
Suppose $\sC$ is a model category.
Then:
\begin{enumerate}
\item ${\rm Cof} \subset {\rm Mor}(\sC)$ is closed under cobase change.
\item ${\rm Cof} \cap \sW \subset {\rm Mor}(\sC) $ is closed under cobase change.
\item ${\rm Fib} \subset {\rm Mor}(\sC)$ is closed under base change.
\item ${\rm Fib} \cap \sW \subset {\rm Mor}(\sC)$ is closed under base change. 
\end{enumerate}
\end{prop}
\begin{proof}
We prove (1), the rest are similar. 
Let $f : X \to Y$ be a cofibration and $g : W \to Z$ be a cobase change of $f$. 
By the axioms of a model category, it is enough to show that $g$ has the LLP with respect to acyclic fibrations. 
Indeed, suppose $p : A \to B$ is an acyclic fibration and consider a diagram
\[
\begin{tikzcd}
W \ar[d,"g"] \ar[r] & A \ar[d,"p"] \\
Z \ar[r] & B .
\end{tikzcd}
\]
We want to produce a lift $h : Z \to A$.
Amending this diagram to the left with $f$ we have a diagram
\[
\begin{tikzcd}
X \ar[r] \ar[d, "f"] & W \ar[r] & A \ar[d,"p"] \\
Y \ar[urr, dotted] \ar[r] &  Z \ar[r] & B .
\end{tikzcd}
\]
The dotted arrow lift exists since $f$ is a cofibration. 
By the universal property of pushouts, the dotted arrow defines a map $h : Z \to A$ lifting $g$ against $p$ as desired. 
\end{proof} 



\section{Localization for model categories}

Localization is a general notion one can define for any category with weak equivalences. 
For model categories, we see that localizations exists in a constructive way. 
The key idea is that one can use the concept of homotopy equivalence, which makes sense in a general model category.
We met the definition of homotopy equivalence for general maps in a model category in the last lecture. 
We begin by exhibiting a class of objects for which homotopy equivalences are particularly well-behaved. 

\subsection{(Co)Fibrancy}

Throughout, $\sC$ is a model category. 

\begin{dfn}
An object $X$ in a model category $\sC$ is called {\bf cofibrant} if the unique map $0 \to X$ is a cofibration. 
An object $X$ in a model category $\sC$ is called {\bf fibrant} if the unique map $X \to \star$ is a fibration. 
\end{dfn}

\begin{rmk}
Every object $X$ admits a {\em cofibrant replacement}
\[
\begin{tikzcd}
0 \ar[rr] \ar[dr, "{\rm Cof}"'] & & X \\
& Q X \ar[ur, "\simeq", "{\rm Fib} \cap \sW"'] & 
\end{tikzcd}
\] 
and a {\em fibrant replacement}:
\[
\begin{tikzcd}
X \ar[rr] \ar[dr, "\simeq", "{\rm Cof} \cap \sW"'] & & \star \\
& R X \ar[ur, "{\rm Fib}"'] & .
\end{tikzcd}
\] 
\end{rmk}

When we restrict ourselves to maps {\em from} a cofibrant object, we see that left homotopy equivalence is actually an equivalence relation!

\begin{prop}
Suppose $X$ is cofibrant. 
Then, {\em left} homotopy equivalence $\sim_L$ is an equivalence relation on the set $\Hom_{\sC}(X,Y)$. 
\end{prop}

Before proving this, we prove the following lemma. 
\begin{lem}
Suppose $X$ is cofibrant, and let $\Cyl(X)$ be a cylinder object. 
Then, the maps
\[
i_1, i_2 : X \xto{j_1,j_2} X \sqcup X \to \Cyl(X)
\]
are both acyclic cofibrations. 
\end{lem}
\begin{proof}
Let's prove this for $i_1$. 
By assumption, we can factor $\id_X : X \to X$ as
\[
X \xto{i_1} \Cyl(X) \xto{\simeq} X .
\] 
By two out of three, we know that $i$ must be a weak equivalence. (This does not yet use cofibrancy.)

Now, $X \sqcup X$ is a coproduct, so it fits in a pushout diagram
\[
\begin{tikzcd}
0 \ar[r] \ar[d] & X \ar[d, "j_1"] \\
X \ar[r,"j_2"] & X \sqcup X .
\end{tikzcd}
\]
By assumption $0 \to X$ is a cofibration.
Since cofibrations are closed under cobase change, we conclude that $j_1 : X \to X \sqcup X$ is a cofibration. 
Finally, cofibrations are closed under composition, we see that $i_1$ is a cofibration. 
\end{proof}

Now on to the proof of the proposition. 

\begin{proof}
Since $X$ is cofibrant, we can take $A$ to be its own cylinder object ${\rm Cyl}(X) = X$. 
Then, we see that any map $f : X \to Y$ is left homotopy equivalent to itself using homotopy $f : \Cyl(X) = X \to X$. 
Thus $f \sim_L f$ via $f$. 

Now, we want to see that $f \sim_L g$ then $g \sim_L f$. 
If $\sigma : X \sqcup X \to X \sqcup X$ is the flip map, then we have $(f + g) \circ \sigma = g + f$. 

Finally, suppose $f \sim_L g$ and $g \sim_L h$ with left homotopies $H : \Cyl(X) \to X$, $H' : \Cyl(X)' \to X$. 
Then, consider the pushout diagram
\[
\begin{tikzcd}
X \ar[r,"i_0", "\simeq"'] \ar[d, "i_1'"', "\simeq"] & \Cyl(X) \ar[d] \\
\Cyl(X)' \ar[r] & Z
\end{tikzcd}
\]
By the lemma above, we know that $i_0, i_1'$ are acyclic cofibrations. 
Thus, by the universal property of pushouts it is immediate to see that $Z$ itself is a cylinder object for $X$, $Z = \Cyl(X)''$ and the induced map $H'' : \Cyl(X)''\to X$ is a left homotopy $f \sim_L h$. 

\end{proof}

A very similar proof shows the dual notion.

\begin{prop}
Suppose $Y$ is fibrant. 
Then $\sim_R$ is an equivalence relation on ${\rm Hom}_{\sC}(X,Y)$. 
In fact, if $X$ is also cofibrant, then $f,g \in {\rm Hom}_\sC(X,Y)$ satisfy
\[
f \sim_L g \iff f \sim_R g .
\]
\end{prop}

Thus, when we look at the full subcategory of objects that are both fibrant and cofibrant 
\[
\sC^{\circ} \subset \sC 
\]
the relations $\sim_L,\sim_R$ are equivalent equivalence relations.

\begin{dfn}
The {\bf homotopy category} $\ho \sC$ of a model category $\sC$ is the subcategory of $\sC$ consisting of objects that are fibrant and cofibrant (so the same objects as $\sC^{\circ}$), and morphisms are
\[
\Hom_{\ho \sC} (X,Y) = \Hom_{\sC}(X,Y) / \sim
\]
where $\sim$ is either $\sim_L$ or $\sim_R$. 
\end{dfn}

\subsection{Back to localization}

%
%Note that there are natural functors
%\[
%\sC^{\circ} \to \sC \to \ho \sC .
%\]
%The first is just the inclusion of the full subcategory of fibrant-cofibrant objects. 
%The next is the quotient map

We go back to the notion of localization of a category with weak equivalences. 
Given any model category $\sC$, we will see that $\ho \sC$ is a model for its localization. 

\begin{prop}
[Whitehead]
Suppose $\sC$ is a model category and $X,Y$ are both fibrant and cofibrant.
Then, $f : X \xto{\simeq} Y$ is a weak equivalence if and only if $f$ is a (left or right) homotopy equivalence. 
\end{prop}

\begin{proof}
First, the forward direction. 
By the factorization axioms of a model category we can factor any map as an acyclic cofibration followed by a fibration. 
If $f$ is a weak equivalence, by the two out of three property we see that $f$ actually factors as an acyclic cofibration followed by an {\em acyclic} fibration. 
\[
\begin{tikzcd}
X \ar[rr,"f"] \ar[dr, "{\rm Cof} \cap \sW"', "g"] & & Y \\
& Z \ar[ur, "{\rm Fib} \cap \sW"', "h"] & .
\end{tikzcd}
\]
Since $X,Y$ are both fibrant and cofibrant, we see that $Z$ is also both fibrant and cofibrant. 
Since the composition of homotopy equivalences is a homotopy equivalence, it suffices to show that both $g,h$ are homotopy equivalences. 

We consider the acyclic fibration $h : Z \to Y$, where both $Z,Y$ are fibrant cofibrant.
The proof for the acyclic cofibration $g : X \to Z$ is completely similar.
By the LLP of cofibrations against acyclic fibrations, there exists a dotted map $\Tilde{h}^{-1}$ in the diagram
\[
\begin{tikzcd}
0 \ar[r] \ar[d,"{\rm Cof}"'] & Z \ar[d, "h"] \\
Z \ar[r, equals] \ar[ur, dotted, "\Tilde{h}^{-1}"] & Z .
\end{tikzcd}
\]
Note that $\Tilde{h}^{-1}$ is an ordinary right inverse to $h$, $h \circ \Tilde{h}^{-1} = \id_Z$.
We want show that $\Tilde{h}^{-1} \circ h \sim_L \id_Z$. 

Let $\Cyl(Z)$ be a cylinder object for $Z$
\[
Z \sqcup Z \to \Cyl(Z) \xto{r} Z
\] 
and consider the commuting diagram
\[
\begin{tikzcd}
Z \sqcup Z \ar[d] \ar[r, "\Tilde{h}^{-1} \circ h + \id_Z"] & Z \ar[d,"h"] \\
\Cyl(Z) \ar[ur, dotted] \ar[r, "h \circ r"] & Y .
\end{tikzcd}
\]
Note the diagram commutes since $\Tilde{h}^{-1}$ is a right inverse to $h$. 
Again, by the LLP of cofibrations against acyclic fibrations, the dotted map above exists. 
One immediately checks that this is the desired homotopy. 

Now, for the converse. 

\end{proof}

This proposition implies the following result of Whitehead. 

\begin{thm}[Whitehead]
Let $\sC$ be a model category. 
Then, there is an equivalence of categories
\[
\sC[\sW^{-1}] \xto{\cong} \ho \sC .
\]
\end{thm}

\subsection{Another model for the homotopy category}

There is another description of the homotopy category of a model category. 

We have already mentioned how every object $X \in \sC$ admits a, non-unique, fibrant and cofibrant replacement that we denote by $RX$ and $QX$, respectively. 
Although these are not unique replacements when viewed in the whole category, they do descend to functors
\begin{align*}
R : & \sC \to \sC^{fib} / \sim_R \\
Q : & \sC \to \sC^{cof} / \sim_L .
\end{align*}
Here, $\sC^{fib} / \sim_R$ is the category of fibrant objects with morphisms given by $\sim_R$-equivalence classes.
Similarly for $\sC^{cof}$. 
If we restrict $R$ to the full subcategory of {\em cofibrant} objects, we see that it induces a functor
\[
R |_{cof} : \sC^{cof} \to \sC^{\circ}/\sim = \ho \sC .
\]
Similarly, the functor $Q$ restricted to $\sC^{fib}$ determines a functor
\[
Q |_{fib} : \sC^{fib} \to \sC^{\circ}/\sim = \ho \sC .
\]

\begin{prop}
The homotopy category $\ho \sC$ is equivalent to the category consisting of {\bf all} objects of $\sC$, and whose morphisms between $X, Y \in \sC$ are
\[
\Hom_{\ho \sC}(R|_{cof} Q X, R|_{cof} Q X) .
\]
The equivalence is induced by the natural functor $\sC \to \ho \sC$. 
\end{prop}

\begin{rmk}
We could have also used $\Hom_{\ho \sC}(Q|_{fib} R X, Q|_{fib} R X)$ and this would give yet another equivalent category to $\ho \sC$. 
\end{rmk}

\section{Model structure on $\dgVect_k$}

There is a model structure on the category $\sC = \dgVect_k$ of dg vector spaces over a field $k$ where:
\begin{itemize}
\item The weak equivalences are the quasi-isomorphisms.
That is, maps of dg vector spaces $f : V \to W$ inducing and isomorphism $H^*f : H^*V \to H^*W$. 
\item The cofibrations are maps of dg vector spaces $f : V \to W$ that are degreewise injective maps of vector spaces:
\[
f^n : V^n \hookrightarrow W^n .
\]
\item The fibrations are maps of dg vector spaces $f : V \to W$ that are degreewise surjective maps of vector spaces:
\[
f^n : V^n \twoheadrightarrow W^n .
\] 
\end{itemize}

\begin{thm}[Hovey]
The choices above make $\dgVect_k$ into a cofibrantly generated model category.
\end{thm}

Roughly, cofibrantly generated means that there is a set cofibrations that generate all other cofibrations. 
We will come back to the concept later. 

\section{Adjunctions}

A natural question is when a functor $F : \sC \to \sD$ between model categories descends to a functor between the associated homotopy categories. 
When $F$ is a {\em homotopy invariant functor}, meaning it preserves weak equivalences, then it does indeed descend 
\[
\Bar{F} : \ho \sC \to \ho \sD .
\]
The problem is, homotopy invariance is a quite a strong condition, and many functors we run into in practice do not satisfy the condition. 

To answer the question in more generality, we introduce the concept of a {\em derived functor}.
First, the following general idea.
\begin{dfn}
A {\bf left Kan extension} of a functor $F : \sC \to \sD$ along a functor $G : \sC \to \sE$ is a pair of a functor $\Lan_G(F) : \sE \to \sD$ fitting into the diagram
\[
\begin{tikzcd}
\sC \ar[d, "G"'] \ar[r,"F"] & \sD \\
\sE \ar[ur, dotted, "\Lan_G(F)"'] &
\end{tikzcd}
\]
together with a natural transformation $\eta_L : F \to \Lan_G(F) \circ G$ that is {\em initial} among all such pairs. 

A {\bf right Kan extension} of $F$ along $G$ is a pair of a functor $\Ran_G(F) : \sE \to \sD$ fitting into the diagram
\[
\begin{tikzcd}
\sC \ar[d, "G"'] \ar[r,"F"] & \sD \\
\sE \ar[ur, dotted, "\Ran_G(F)"'] &
\end{tikzcd}
\]
together with a natural transformation $\eta_R : \Ran_G(F) \circ G \to F$ that is {\em final} among all such pairs. 
\end{dfn}

When Kan extensions exists they are unique by their universal properties. 

\begin{dfn}
Let $\sC$ be a model category, $\sD$ be any category, and $F : \sC \to \sD$ a functor. 
A {\bf left derived functor} $\LL ' F$ of $F$ is a {\em right} Kan extension of $F$ along the $\sC$-localization $\Gamma_\sC : \sC \to \ho (\sC)$:
\[
\LL ' F = \Ran_{\Gamma_\sC} (F) : \ho \sC \to \sD .
\] 
A {\bf right derived functor} $\RR ' F$ of $F$ is a {\em left} Kan extension
\[
\RR ' F = \Lan_{\Gamma_{\sC}} (G) : \ho \sC \to \sD .
\] 
\end{dfn}

When both $\sC$ and $\sD$ are model categories, there is a notion of a {\em total} derived functor.

\begin{dfn}
Let $\sC, \sD$ be model categories, and $F : \sC \to \sD$ a functor. 
A {\bf total left derived functor} $\LL F$ of $F$ is a {\em right} Kan extension of the composition $\Gamma_\sD \circ F$ along the $\sC$-localization $\Gamma_\sC : \sC \to \ho (\sC)$:
\[
\LL F = \Ran_{\Gamma_\sC} (\Gamma_{\sD} \circ F) : \ho \sC \to \ho\sD .
\] 
A {\bf right derived functor} $\RR F$ of $F$ is a {\em left} Kan extension 
\[
\RR F = \Lan_{\Gamma_{\sC}} (\Gamma_{\sD} \circ G) : \ho \sC \to \ho \sD .
\] 
\end{dfn}

Recall, we say a functor $F : \sC \to \sD$ is {\em left adjoint} to a functor $U : \sD \to \sC$ if there exists a natural isomorphism of bifunctors $\sC \times \sD \to \Set$
\[
\Hom_{\sD}(F(-), -) \to \Hom_{\sC}(-, U(-)) .
\]
Thus, for each $X \in \sC$ and $Y \in \sD$ we have
\[
\Hom_{\sD}(F(X), Y) \xto{\cong} \Hom_{\sC}(X, U(Y)) .
\]
In symbols, we write $F \dashv G$.

\begin{dfn}
An adjunction between model categories
\[
\begin{tikzcd}
\sC\ar[r,bend left,"F",""{name=A, below}] & \sD\ar[l,bend left,"U",""{name=B,above}] \ar[from=A, to=B, symbol=\dashv]
\end{tikzcd}
\]
is a {\bf Quillen adjunction} if $F$ preserves cofibrations and acyclic cofibrations. 
\end{dfn}

\begin{rmk}
Equivalently, a Quillen adjunction is an adjunction as above such that $U$ that preserves fibrations and acyclic fibrations. 
For instance, note that all of the data in the diagram 
\[
\begin{tikzcd}
FX \ar[r] \ar[d, "{\rm Cof} \cap \sW"'] & Y \ar[d, "{\rm Fib}"] \\
FX' \ar[r] \ar[ur, dotted] & Y' 
\end{tikzcd}
\]
is equivalent, by the adjunction, to the diagram
\[
\begin{tikzcd}
X \ar[r] \ar[d, "{\rm Cof} \cap \sW"'] & U Y \ar[d, "{\rm Fib}"] \\
X' \ar[r] \ar[ur, dotted] & U Y' .
\end{tikzcd}
\]
\end{rmk}

\begin{thm}[Quillen]
If $F\dashv U$ is a Quillen adjunction, then $\LL F$ and $\RR U$ exist and define an adjoint pair between homotopy categories
\[
\begin{tikzcd}
\ho \sC\ar[r,bend left,"\LL F",""{name=A, below}] & \ho \sD\ar[l,bend left,"\RR U",""{name=B,above}] \ar[from=A, to=B, symbol=\dashv]
\end{tikzcd}
\]
Furthermore, $\LL F$ and $\RR U$ are computed by formulas
\begin{align*}
\LL F (X) & = \Gamma_\sD F(Q X) \\
\RR G (Y) & = \Gamma_\sC G(R Y) .
\end{align*}
\end{thm}

\begin{rmk}
It is not necessary to assume $F,U$ participate in a Quillen adjunction to ensure that $\LL F, \RR U$ exist. 
However, their formulas may not be as simple as in the theorem. 
\end{rmk}

\begin{dfn}
A {\bf Quillen equivalence} is a Quillen pair $F \dashv U$ such that $\LL F \dashv \RR U$ is an equivalence of respective homotopy categories.
\end{dfn}

\newcommand{\verteq}{\rotatebox{90}{$\,\cong$}}

\begin{thm}[Quillen]
Suppose $F \dashv U$ is a Quillen pair. 
If, for every cofibrant object $X \in \sC^{cof}$ and fibrant object $Y \in \sC^{fib}$ one has
\[
(f : X \to U Y) \in \sW_\sC \iff (\eta f : F X \to Y) \in \sW_\sD
\]
then $\LL F$ (and $\RR G$) defines an equivalence of categories
\[
\begin{tikzcd}
\ho \sC\ar[r,bend left,"\LL F",""{name=A, below}] & \ho \sD\ar[l,bend left,"\RR U",""{name=B,above}] \ar[from=A, to=B, symbol=\verteq] .
\end{tikzcd}
\]
\end{thm}


\end{document}




\end{document}