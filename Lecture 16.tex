\documentclass[11pt]{amsart}

\usepackage{macros}

\linespread{1.25}

\usepackage[final]{pdfpages}

\setcounter{tocdepth}{2}

\def\Fun{{\sf Fun}}
\def\Top{{\sf Top}}
\def\colim{{\rm colim}\;}
\def\Sing{{\rm Sing}}
\def\Cat{{\sf Cat}}
\def\Spec{{\rm Spec}}
\def\lcone{\triangleleft}
\def\rcone{\triangleright}


\title{Lecture 15: Small algebras}

\begin{document}
\maketitle

\section{Excursion: (co)limits in $\infty$-categories}

We have so far avoided talking about (co)limits in the $\infty$-categorical sense. 
We cannot avoid this anymore. 

In practice, defining categorical construction, such as (co)limits, at the level of $\infty$-categories is completely analogous to the ordinary categorical versions. 
The main advantage, is that $\infty$-category is constructed in such a way that these constructions play well with homotopy equivalence.
Thus, from a model categorical perspective, they behave a lot like homotopy (co)limits. 

\subsection{A reminder on ordinary (co)limts}

We take a brief aside to remind ourselves about ordinary (co)limits, and state a potentially unfamiliar axiomatic way of defining at them. 

So, let $F : \sI \to \sC$ be a functor between ordinary categories. 
The most standard definition of a colimit of $F$ is given by a collection of arrows in $\sC$
\[
\{ \eta_{i} : F(i) \to X \;, \; i \in \sI | F(i \to j) \circ \eta_i = \eta_j \; \forall i \to j\}
\]
that is {\em initial} among all such collections. 
This means that given any other collection of arrows $\{\eta_i' : F(i) \to Y\}$ there is a unique map $f : X \to Y$ compatible with the collections. 
In this situation, we usually say $X$ is the colimit of $F$ and write $\colim_\sI F = X$. 
There is the clear dual notion for limits. 

In the last lecture, we have introduced the {\em join} of two categories $\sC \star \sD$. 
Also, recall the ``singleton category", $[0]$ which consists of a single object $0$ and no non-identity maps. 
For any ordinary category $\sI$, define the two categories
\[
\begin{array}{ccccc}
({\rm Left \; cone}) & \sI^{\lcone} & := & [0] \star \sI \\
({\rm Right \; cone}) & \sI^{\rcone} & := & \sI \star [0]
\end{array}
\]
Note that in both cases, there are natural functors $\sI \to \sI^{\lcone} , \sI^{\rcone}$. 

Last time we recalled the over/under category of category over / under an object. 
Here is a more general construction. 

\begin{dfn}
Let $F : \sI \to \sC$ be a functor. 
\begin{itemize}
\item[(1)]
Define the {\bf over category} $\sC_{/F}$ to be the full subcategory
\[
\sC_{/F} \subset \Fun(\sI^{\lcone}, \sC)
\]
consisting of functors $G : \sI^{\rcone} \to \sC$ such that the composition
\[
\sI \to \sI^{\rcone} \xto{G} \sC
\]
is $F$. 

\item[(2)]
Define the under category $\sC_{F /}$ to be the full subcategory
\[
\sC_{F / } \subset \Fun(\sI^{\rcone}, \sC)
\]
consisting of functors $G : \sI^{\rcone} \to \sC$ such that the composition
\[
\sI \to \sI^{\rcone} \xto{G} \sC
\]
is $F$. 
\end{itemize}
\end{dfn}

Let $F : \sI \to \sC$ be a functor.
Note that any object $X \in \sC$ determines a functor
\[
X_F : \sI^{\lcone} \to \sC
\]
which sends an object $j \in \sI$ to $F(j)$ and $[0]$ to $X$. 
Clearly, then, this determines an element $X_F \in \sC_{F /}$. 
Similarly for the right cone and the overcategory. 

The following is a tautological observation. 
\begin{lem}
The object $X \in \sC$ is a {\em colimit} of $F : \sI \to \sC$ if and only if $X_F \in \sC_{F /}$ is an {\em initial} object.
Similarly, $X \in \sC$ is a limit of $F : \sI \to \sC$ if and only if $X_F \in \sC_{/F}$ is a {\em final} object. 
\end{lem}

This means that we can define limits / colimits using the universal property of being final / initial in an appropriate diagram category!
This observation is actually most of the work in defining the notion of a (co)limit in an $\infty$-category.
The only thing we are missing is the appropriate notion of initial and final.

\subsection{Initial / final objects}

For an ordinary category, an object $0 \in \sC$ is {\em initial} if $\Hom(0, X) = \{\star\}$ is the singleton set for any $X \in \sC$. 
For the appropriate notion in $\infty$-categories it is then fairly clear what we must say: an object $0 \in \sC$ is initial if the {\em space} of maps from $0$ to any other object is {\em contractible}. 

\begin{dfn}
Let $\sC$ be an $\infty$-category.
An object $0 \in \sC$ is {\bf initial} if 
\[
\Map_{\sC}(0, X)
\]
is contractible for all $X \in \sC$.
An object $\star \in \sC$ is {\bf final} if 
\[
\Map_{\sC}(X, \star)
\]
is contractible for all $X \in \sC$. 
\end{dfn}

\subsection{(Co)limits}

Finally, we can define (co)limits. 
If $\sI$ is any $\infty$-category, define the auxiliary $\infty$-categories
\[
\begin{array}{ccccc}
({\rm Left \; cone}) & \sI^{\lcone} & := & \Delta^0 \star \sI \\
({\rm Right \; cone}) & \sI^{\rcone} & := & \sI \star \Delta^0
\end{array}
\]
just as before. 
(Recall, we have made sense of what it means to take the join of two simplicial sets). 
We can then define the over / under infinity categories $\sC_{/F} \subset \Fun(\sI^{\lcone}, \sC)$ and $\sC_{F/} \subset \Fun(\sI^{\rcone}, \sC)$ just as in the ordinary way. 
Here, of course, by functors between $\infty$-categories we mean the simplicial set of maps, which is again an $\infty$-category. 
The only non-obvious thing to verify is why these simplicial sets still satisfy the weak Kan condition, but we leave it as an exercise. 

\begin{dfn}
Let $F : \sI \to \sC$ be a functor of $\infty$-categories.
\begin{itemize}
\item[(1)] We say $X \in \sC_{/F}$ is a {\em limit} of $F$ if $X$ is a final object in $\sC_{/F}$. 
\item[(2)] We say $X \in \sC_{F/}$ is a {\em colimit} of $F$ if $X$ is an initial object in $\sC_{F/}$. 
\end{itemize}
\end{dfn}

\subsection{The relation to model categories}

For $\infty$-categories that arise from model categories through Dwyer-Kan localization, there is a beautiful relationship between (co)limits in $\infty$-categories and the corresponding {\em homotopy} (co)limits at the model category level.

Dwyer-Kan localization produces a simplicial category, which is at the level that this theorem is stated. 
Note that given any functor $F : \sI \to \sC$ of simplicial categories, there is an induced functor on the simplicial nerve.
This means, we have a functor of $\infty$-categories
\[
N_{\Delta} (F) : N_{\Delta}(\sI) \to N_{\Delta}(\sC) .
\]

\begin{thm}
Suppose $F : \sI \to \sC$ is a functor of simplicial categories.
Suppose 
\[
\{\eta_i : F(i) \to X \; | \; i \in \sI\}
\]
is some collection of arrows in $\sC$. 
The following conditions are equivalent:
\begin{itemize}
\item 
The collection $\{\eta_i\}$ exhibits $X \in \sC$ as the homotopy colimit of $F$;
\item 
The collection $\{\eta_i\}$ defines an extension of $N_{\Delta} (F)$ to a functor
\[
N_{\Delta} (F)_{\eta} : N_{\Delta}(\sI)^{\rcone} \to N_{\Delta}(\sC) 
\]
which is an initial object in $N_{\Delta}(\sC)_{N_{\Delta}(F) / }$. 
In other words, $X$ is a colimit of $N_{\Delta}(F)$, in the $\infty$-categorical sense.
\end{itemize}
There is a completely similar statement for limits.
\end{thm}

\section{Small algebras}




 
\end{document}