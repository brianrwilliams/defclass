\documentclass[11pt]{amsart}

\usepackage{macros}

\linespread{1.25}

\usepackage[final]{pdfpages}

\setcounter{tocdepth}{2}

\def\Fun{{\sf Fun}}
\def\Top{{\sf Top}}
\def\colim{{\rm colim}\;}
\def\Sing{{\rm Sing}}
\def\Cat{{\sf Cat}}

\title{Lecture 13: Transferring model structures}

\begin{document}
\maketitle

In this lecture we have two goals: 
\begin{itemize}
\item[(1)] recall the standard model structure on the category of dg vector spaces, and transfer it to model category structures on dg Lie algebras and commutative dg algebras;
\item[(2)] formulate a variant of the nerve construction that produces an $\infty$-category from an ordinary model category. 
\end{itemize}

\section{Model structures on algebras}

\begin{thm}
\label{thm: trans}
Suppose
\[
\begin{tikzcd}
\sC\ar[r,bend left,"F",""{name=A, below}] & \sD\ar[l,bend left,"G",""{name=B,above}] \ar[from=A, to=B, symbol=\dashv]
\end{tikzcd}
\]
is an adjoint pair and that $\sC$ is a cofbrantly generated model category. 
Suppose further that
\begin{itemize}
\item[(1)] the functor $G$ commutes with sequential colimits;
\item[(2)] a map in $\sD$ that has the LLP with respect to $G^{-1}({\rm Fib}_{\sC}) \subset {\rm Mor}(\sD)$ is in $G^{-1}(\sW)$. 
\end{itemize}
Then, $\sD$ has the structure of a cofibrantly generated model category where 
\begin{itemize}
\item the sets
\[
\{F(i) \; | \; i \in {\rm Cof}_\sC\} \;\; , \;\; \{F(j) \; | \; j \in {\rm Cof}_{\sC} \cap \sW_{\sC} \}
\]
generated the cofibrations and acyclic cofibrations respectively;
\item $f \in {\rm Mor}(\sD)$ is a fibration / weak equivalence if and only if $G(f)$ is a fibration / weak equivalence. 
\end{itemize}
\end{thm}

\begin{proof} (Sketch)
The main thing to verify is that every morphism $f : X \to Y$ in $\sD$ factors as
\[
X \xto{i} Z \xto{p} Y
\]
where $j$ is a cofibration and $p$ is an acyclic fibration. 
A major part of the proof involves a diagram in $\sC$ of the form



We can state condition (2) in the theorem as follows. 
First, call $f \in {\rm Mor}(\sD)$ a fibration / weak equivalence if and only if $G(f)$ is a fibration / weak equivalence in $\sC$. 
Then, (2) requires that if a map in $\sD$ has the LLP with respect to all fibrations in $\sD$ then its a weak equivalence. 

This condition is quite hard to verify in practice, but thankfully there is an easier condition one can verify to ensure this holds. 

\begin{prop}\label{prop: easy}
Let $F,G$ be adjoint functors as above where $\sC$ is {\em any} model category. 
Call $f \in {\rm Mor}(\sD)$ a fibration / weak equivalence if and only if $G(f)$ is a fibration / weak equivalence in $\sC$. 
If
\begin{itemize}
\item[(a)] every object in $\sD$ is fibrant;
\item[(b)] every object in $\sD$ has a path object;
\end{itemize}
then a map that has the LLP with respect to every fibration is a weak equivalence.
\end{prop}

\begin{rmk}
The path object is the dual notion to a cylinder object.
Recall, if $X$ is an object of a model category, a path object is an object $PX$ together with maps $s : X \to PX$, $(p_0,p_1) : PX \to X$ such that $s$ is a weak equivalence, $p_0 \times p_1$ is a fibration, and the diagram
\[
\begin{tikzcd}
X \ar[r,"s"', "\simeq"]  \ar[dr, "1 \times 1"'] & P X \ar[d, "p_0 \times p_1"] \\
& X \times X .
\end{tikzcd}
\]
commutes.
\end{rmk}

\section{Transferring from $\dgVect$}

First of all, there is a natural model structure on the category of dg vector spaces given by the following:
\begin{itemize}
\item a map $f : V \to W$ in $\dgVect_k$ is a fibration if an only if $f : V^n \to W^n$ is surjective for each $n$;
\item a map $f : V \to W$ in $\dgVect_k$ is a cofibration if and only if $f : V^n \to W^n$ is injective for each $n$;
\item a map $f : V \to W$ in $\dgVect_k$ is a weak equivalence if and only if $f$ is a quasi-isomorphism. 
\end{itemize}

\begin{ex}
Show that this is a model structure on $\dgVect_k$.
What are the fibrant / cofibrant objects?
\end{ex}

\begin{thm}[\cite{Hov}]
This model structure is a cofibrantly generated model structure on $\dgVect_k$. 
\end{thm}

\begin{rmk}
Suppose that $W$ is an ordinary vector space.
Define the dg vector space $D^n(W)$ by 
\[
\fc^n (W) = W[-n] \xto{\id_W} W[-n-1] .
\]
We see that $\fc^n(W)$ is acyclic and that there is a natural inclusion of dg vector spaces $W[-n] \to \fc^n(W)$.
Note that $\fc^n(W)$ is precisely the cone of the map of dg vector spaces $\id : W[-n] \to W[-n]$. 

The set of all inclusions
\[
\{k[-n] \to \fc^n(k)\}_{n \in \ZZ}
\]
is a generating set for the cofibrations above. 
The set of maps
\[
\{0 \to \fc^n(k)\}_{n \in \ZZ}
\]
is a generating set for the acyclic cofibrations. 
\end{rmk}

\subsection{The model structure on $\dgCAlg$}
\def\oblv{{\rm oblv}}
\def\Tens{{\rm Tens}}

We use Theorem \ref{thm: trans} to induce a model structure on commutative dg algebras in the following way. 
Recall the adjunction
\[
\begin{tikzcd}
\dgVect_k \ar[r,bend left,"\Sym(-)",""{name=A, below}] & \dgCAlg_k \ar[l,bend left,"{\rm oblv}_C",""{name=B,above}] \ar[from=A, to=B, symbol=\dashv]
\end{tikzcd}
\]
Where $\Sym(-) : V \mapsto \Sym(V)$ is the symmetric algebra functor, and $\oblv_C$ is the forgetful functor. 

In order to apply Theorem \ref{thm: trans}, we need to verify that the following two conditions are satisfied:
\begin{itemize}
\item[(1)] The functor $\oblv_C$ preserves sequential colimits;
\item[(2)] Call a map $f : A \to B$ of commutative dg algebras is a fibration if $\oblv_C(f)$ is a fibration. 
Then, we need to show that if a map of commutative dg algebras has the LLP with respect to every fibration, then it is a quasi-isomorphism.
\end{itemize}

Item (1) follows from the stronger result:

\begin{lem}
The forgetful functor 
\[
\oblv_C : \dgCAlg_k \to \dgVect_k
\]
creates sifted colimits.
\end{lem}

To show (2) we appeal to Proposition \ref{prop: easy}. 
The first condition in Proposition \ref{prop: easy} is immediate since every dg vector space is fibrant. 
So, all we need to show is the following. 

\begin{lem}
Every object in $\dgCAlg_k$ has a path object.
\end{lem}
\begin{proof}
Let $A$ be a commutative dg algebra. 
Define
\[
M = \bigoplus_{n \in \ZZ} \bigoplus_{(a_1,a_2) \in (A \times A)_n} \fc^n(k) .
\]
This looks complicated, but its just a huge acyclic object that we will construct the path space out of. 
For each $(a_1,a_2) \in (A \times A)_n$ we have a map of dg vector spaces
\[
p_{(a_1,a_2)} : \fc^n(k) \to A \times A
\]
that sends the generator in degree $n$, $1 \in \fc^n(k)^n = k$ to $(a_0,a_1)$, and the generator in degree $n+1$, $1 \in \fc^n(k)^{n+1}$ to $\d_{A \times A} (a_0, a_1)$. 
By the universal property of direct sums, the collection of all of these define a map
\[
p : M \to A \times A
\]
and by the universal property of the symmetric algebra we have a map of commutative algebras
\[
\Sym(p) : \Sym(M) \to A \times A .
\]
Note that $\Sym(p)$ is degreewise surjective by construction. 

In any symmetric algebra we have a unit $1 \in \Sym^0(M) = k \subset \Sym(M)$. 
Compiling these facts we obtain a commutative diagram
\[
\begin{tikzcd}
A \ar[r, "\id \tensor 1"] \ar[dr, "\id \times \id"'] & A \tensor_k \Sym(M) \ar[d, "(\id \times \id) \tensor \Sym(p)"] \\
& A \times A .
\end{tikzcd}
\]
By construction, $\Sym(M)$ is acyclic, so the top map is a quasi-isomorphism. 
Also, the vertical right map is a fibration since $\Sym(p)$ is. 
\end{proof}

To summarize, we have endowed $\dgCAlg_k$ with a cofibrantly generated model category structure with the properties:
\begin{itemize}
\item[(1)] A map $f : A \to B$ of commutative dg algebras is a weak equivalence if and only if $\oblv_C(f) : \oblv_C(A) \to \oblv_C(B)$ is a weak equivalence in $\dgVect_k$.
That is, a quasi-isomorphism. 
\item[(2)] A map $f : A \to B$ of commutative dg algebras is a fibration if and only if $\oblv_C(f) : \oblv_C(A) \to \oblv_C(B)$ is a fibration in $\dgVect_k$.
That is, a degreewise surjection. 
\item[(3)] The set of maps
\[
\{{\rm Tens}_L(k[n]) \to {\rm Tens}_L (\fc^n(k))\}_{n \in \ZZ}
\]
generate the cofibrations. 
\end{itemize}


\subsection{The model structure on $\dgLie$}

We can also utilize the transfer theorem to endow $\dgLie_k$ with a model structure by appealing to the adjunction
\[
\begin{tikzcd}
\dgVect_k \ar[r,bend left,"{\rm free}_L(-)",""{name=A, below}] & \dgLie_k \ar[l,bend left,"{\rm oblv}_L",""{name=B,above}] \ar[from=A, to=B, symbol=\dashv]
\end{tikzcd}
\]
between the free Lie algebra functor and the forgetful functor from Lie algebras to vector spaces. 

The proof is very similar, so we just read off the properties of this model structure on $\dgLie_k$. 

\begin{itemize}
\item[(1)] A map $f : \fg \to \fh$ of dg Lie algebras is a weak equivalence if and only if $\oblv_L(f) : \oblv_L(\fg) \to \oblv_L(\fh)$ is a weak equivalence in $\dgVect_k$.
That is, a quasi-isomorphism. 
\item[(2)] A map $f : \fg \to \fh$ of dg Lie algebras is a fibration if and only if $\oblv_L(f), \oblv_L(\fh)$ is a fibration in $\dgVect_k$.
That is, a degreewise surjection. 
\item[(3)] The set of maps
\[
\{{\rm free}_L(k[n]) \to {\rm free}_L (\fc^n(k))\}_{n \in \ZZ}
\]
generate the cofibrations. 
\end{itemize}

\section{From model to $\infty$-categories}

We have already discussed how one can obtain an $\infty$-category both from an ordinary category (the {\em classical} nerve) and a simplicial category (the {\em simplicial} / {\em coherent} nerve). 
In the latter case, the simplicial nerve remembers the actual simpicial model category structure and set up the equivalence between two different theories of $\infty$-categories. 

Now, we would like to discuss a construction of an $\infty$-category from a model category that is sensitive to all of the structures of the model category. 
For instance, the $\infty$-category will detect weak equivalences as ordinary equivalences. 



\end{document}
