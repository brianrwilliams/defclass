\documentclass[11pt]{amsart}

\usepackage{macros}

\linespread{1.25}

\usepackage[final]{pdfpages}

\setcounter{tocdepth}{2}

\title{Lecture 4: Model categories}

\begin{document}

\section{The idea of a model category}
Roughly, the theory of model categories was developed to better handle the notion of a ``homotopy equivalence". 
For us, the fundamental example of a homotopy equivalence is a quasi-isomorphism of dg vector spaces. 

The first, and perhaps most obvious, attempt to encode homotopy, or {\em weak}, equivalences in a category is to prescribe some class of morphisms that ware well-behaved with respect to composition. 
The definition is the following. 

\begin{dfn}
A {\bf category with weak equivalences} is a category $\sC$ together with a set
\[
\sW \subset {\rm Mor}(\sC)
\]
such that
\begin{enumerate}
\item If $f$ is an isomorphism, then $f \in \sW$;
\item if $f,g$ are morphisms such that $f \circ g$ exists then: if two of $f, g, f \circ g$ are in $\sW$ then the third is as well (`two out of three").
\end{enumerate}
\end{dfn}

\section{Localization}

\brian{Whitehead localization of cat with weak equiv. Cof/fib construction for model cats. Why its better behaved for model cats.}

\section{Adjoints and extensions}

\brian{Quillen adjunction, left/right Kan extensions, examples, Quillen equivalence}

\section{Examples}

\section{dgVect, dgLie, etc. Top.}

\end{document}