\documentclass[11pt]{amsart}

\usepackage{macros}

\linespread{1.25}

\usepackage[final]{pdfpages}

\setcounter{tocdepth}{2}

\title{Lecture 4: Model categories}

\def\Fun{{\sf Fun}}
\def\colim{{\rm colim}}
\def\Set{{\sf Set}}
\def\Top{{\sf Top}}

\begin{document}

\section{Some categorical remarks}

\subsection{(Co)Limits}

We recall some general notions in ordinary category theory. 
For a textbook reference see \cite{MacLane}.

Let $\sC$ and $\sI$ be categories. 
For each object $X \in \sC$ let $\ul{X} : \sI \to \sC$ be the functor that sends every object of $\sI$ to $X \in \sC$ and every morphism to the identity. 
This construction extends to a functor
\[
\ul{(-)} : \sC \to \Fun(\sI, \sC) .
\]

\begin{dfn}
Suppose $\sI$ is a small category and let $F : \sI \to \sC$ be a functor. 
A {\bf colimit} of $F$ is an object $X \in \sC$ together with a natural transformation
\[
t : F \to \ul{X}
\]
such that for every $Y \in \sC$ and every natural transformation $s : F \to \ul{Y}$ there exists a unique map $s' : X \to Y$ making $\ul{s'} t = s$. 
\end{dfn}

Any two colimits are naturally isomorphic. 
If the colimit of a functor $F : \sI \to \sC$ exists, we will write it as $\colim_{\sI} F \in \sC$. 

Many familiar categories posses the property that {\em all} colimits exist. 
These include $\Set, \Top$, and $\Vect_k$.
Moreover, colimits are functorial in the natural way. 

\begin{dfn}
A category $\sI$ is called {\bf filtered} if, for any finite category $\sJ$ and functor $J : \sJ \to \sI$, there exists an object $i \in \sI$ and a natural transformation $F \to \ul{i}$. 
\end{dfn}

If $F : \sI \to \sC$ is a functor, $\sI$ is filtered, and $\colim_{\sI} F$ exists, then the colimit is called a {\em filtered colimit}. 
A natural example of a filtered category is the poset
\[
\sI = \ZZ_+ = \{0 \to 1 \to 2 \to \cdots \} .
\]
Resulting colimits are special types of filtered colimits called {\em sequential colimits}. 

\subsubsection{}
The notion of a limit is defined in a dual way. 
The aforementioned categories also admit all limits. 

\section{The idea of a model category}
Roughly, the theory of model categories was developed to better handle the notion of a ``homotopy equivalence". 
For us, the fundamental example of a homotopy equivalence is a quasi-isomorphism of dg vector spaces. 

The first, and perhaps most obvious, attempt to encode homotopy, or {\em weak}, equivalences in a category is to prescribe some class of morphisms that ware well-behaved with respect to composition. 
The definition is the following. 

\begin{dfn}
A {\bf category with weak equivalences} is a category $\sC$ together with a set
\[
\sW \subset {\rm Mor}(\sC)
\]
such that
\begin{enumerate}
\item If $f$ is an isomorphism, then $f \in \sW$;
\item if $f,g$ are morphisms such that $f \circ g$ exists then: if two of $f, g, f \circ g$ are in $\sW$ then the third is as well (`two out of three").
\end{enumerate}
\end{dfn}

\subsection{Model categories}

\begin{dfn}
Let $\sC$ be a category and $K \subset {\rm Mor}(\sC)$ be a subset of morphisms. 
A morphism $f : X \to Y$ in $\sC$ has the {\bf left lifting property} (LLP) with respect to $K$ if for any morphism $g : W \to Z$ in $K$ and solid line diagram
\[
\begin{tikzcd}
X \ar[r] \ar[d,"f"] & W \ar[d,"g"] \\ 
Y \ar[r] \ar[ur, dotted, "\exists h"'] & Z 
\end{tikzcd}
\]
there exists a dotted map $h : Y \to W$ making the total diagram commute. 
Dually, we say $f : X \to Y$ has the {\bf right lifting property} (RLP) with respect to $K$ if
for any morphism $g : W \to Z$ in $K$ and solid line diagram
\[
\begin{tikzcd}
W \ar[r] \ar[d,"g"] & X \ar[d,"f"] \\ 
Z \ar[r] \ar[ur, dotted, "\exists h"'] & Y
\end{tikzcd}
\]
there exists a dotted map $h : Z \to X$ making the total diagram commute. 
\end{dfn}

\begin{dfn}
A model category is a category with weak equivalences $(\sC, \sW)$ together with distinguished classes ${\rm Cof}, {\rm Fib} \subset {\rm Mor}(\sC)$ satisfying the following axioms:
\begin{enumerate}
\item[(1)] ...
\end{enumerate}
\end{dfn}

\subsection{Retracts}

Let $\sC$ be any category.
An object $X \in \sC$ is said to be a retract of $Y \in \sC$ if there exists morphisms $i : X \to Y$ and $r : Y \to X$ such that $r \circ i = \id_X$. 

\begin{eg}
A dg vector space $V$ is a retract of $W$ if and only if there exists a dg vector space $W$ such that $W \cong V \oplus Z$.
In this case, the maps can be taken to be the inclusion $i : V \hookrightarrow V \oplus Z$ and the projection $r : V \oplus Z \to V$. 
\end{eg}

\def\Arr{{\sf Arr}}

Given any category $\sC$ we can define its associated {\em arrow category} $\Arr(\sC)$ as follows. 
The objects are the morphisms ${\rm Mor}(\sC)$. 
A morphism from $f : X \to Y$ to $g : X' \to Y'$ is a commutative square
\[
\begin{tikzcd}
X \ar[r] \ar[d,"f"] & X' \ar[d, "g"] \\
Y \ar[r] & Y' .
\end{tikzcd}
\]

\begin{dfn}
A morphism $f \in {\rm Mor}(\sC)$ is a {\bf retract of a morphism} $g \in {\rm Mor}(\sC)$ if $f$ is a retract of $g$ in the arrow category $\Arr(\sC)$. 
Concretely, this means that we have a commutative diagram
\[
\begin{tikzcd}
X \ar[r] \ar[d,"f"] & X' \ar[d, "g"] \ar[r] & X\ar[d, "f"] \\
Y \ar[r] & Y' \ar[r] & Y 
\end{tikzcd}
\]
such that both horizontal compositions are the identity. 
\end{dfn}

Retracts are well-behaved under many categorical properties.

\begin{lem}
If $g$ is an isomorphism and $f$ is a retract of $g$, then $f$ is an isomorphism. 
\end{lem}

We will also consider retracts in the setting of model categories. 
One technical result we will use quite often is the following. 
 
\begin{prop}
Suppose we have a commuting triangle
\[
\begin{tikzcd}
X \ar[rr,"f"] \ar[dr,"i"] & & Y \\
& Z \ar[ur, "p"] & .
\end{tikzcd}
\]
Then:
\begin{enumerate}
\item[(1)] 
if $f$ has LLP with respect to $p$, then $f$ is a retract of $i$;
\item[(2)] if $f$ has RLP with respect to $i$, then $f$ is a retract of $p$. 
\end{enumerate}
\end{prop}

\section{Localization}

Roughly, a localization of a category with weak equivalences is another category in which all of the weak equivalences are isomorphisms. 
Many such categories might exist, and we look at the {\em initial one} with respect to this property. 

\begin{dfn}
Let $(\sC, \sW)$ be a category with weak equivalences. 
A {\bf localization} of $\sC$ with respect to $\sW$ is a category $\sC[\sW^{-1}]$ equipped with a functor
\[
\Gamma : \sC \to \sC[\sW^{-1}]
\]
such that:
\begin{enumerate}
\item if $f \in \sW$ then $\Gamma(f) \in {\rm Mor}(\sC[\sW^{-1}])$ is an isomorphism;
\item if $F : \sC \to \sD$ is any functor such that $f \in \sW$ implies $F(f)$ is an isomorphism, then there is a unique factorization 
\[
\begin{tikzcd}
\sC \ar[rr, "F"] \ar[dr, "\Gamma"] & & \sD \\
& \sC[\sW^{-1}] \ar[ur, dotted, "\exists !"] & .
\end{tikzcd}
\] 
\end{enumerate}
\end{dfn}
It is immediate to see that when a localization exists it is unique. 

The problem of existence is one issue. 
The other, more serious problem for us, is coming up with a model for the localization when it exists. 
This is one problem that {\em model} categories solve!

\subsection{(Co)Fibrancy}

\begin{dfn}
An object $X$ in a model category $\sC$ is called {\bf cofibrant} if the unique map $0 \to X$ is a cofibration. 
An object $X$ in a model category $\sC$ is called {\bf fibrant} if the unique map $X \to \star$ is a fibration. 
\end{dfn}

Every object $X$ admits a {\em cofibrant replacement}
\[
\begin{tikzcd}
0 \ar[rr] \ar[dr, "{\rm Cof}"'] & & X \\
& Q X \ar[ur, "\simeq", "{\rm Fib} \cap \sW"'] & 
\end{tikzcd}
\] 
and a {\em fibrant replacement}:
\[
\begin{tikzcd}
X \ar[rr] \ar[dr, "\simeq", "{\rm Cof} \cap \sW"'] & & \star \\
& R X \ar[ur, "{\rm Fib}"'] & .
\end{tikzcd}
\] 

%\brian{Whitehead localization of cat with weak equiv. Cof/fib construction for model cats. Why its better behaved for model cats.}

\section{Adjoints and extensions}

%\brian{Quillen adjunction, left/right Kan extensions, examples, Quillen equivalence}

\section{Examples}



\section{dgVect, dgLie, etc. Top.}

\end{document}