\documentclass[11pt]{amsart}

\usepackage{macros}

\linespread{1.25}

\usepackage[final]{pdfpages}

\setcounter{tocdepth}{2}

\def\Spec{{\rm Spec}}

\title{Lecture 2: Deformation problems in algebraic geometry}

\def\brian{\textcolor{blue}{BW: }\textcolor{blue}}

\begin{document}
\maketitle

\section{Some algebraic geometry}

Fix a field $k$ of characteristic zero.
Let $\CAlg_k$ denote the category of commutative, associative, and unital algebras over $k$.
For any $A \in \CAlg_k$ define the functor
\[
\begin{array}{cccc}
h_{{\rm Spec}(A)} : & \CAlg_k & \to & \Set \\
& B & \mapsto & {\rm Hom}_{\CAlg_k}(A,B) 
\end{array}
\]
The assignment $A \mapsto h_{{\rm Spec}(A)}$ defines a functor
\[
h : \CAlg^{op}_k \to {\sf Fun}(\CAlg_k , \Set) .
\]
This is the Yoneda embedding for the category of commutative algebras. 
It is fully faithful, in the sense that $h_{\Spec(A)} \cong h_{\Spec(B)}$ as functors if and only if $A$ and $B$ are isomorphic as algebras. 
Thus, studying commutative $k$-algebras is equivalent to studying the functors $\{h_A \; : \; A \in \CAlg_k\}$. 

\begin{rmk}
The notation $h_{{\rm Spec}(A)}$ is used to indicate that we are representing the {\em affine scheme} ${\rm Spec}(A)$ through its functor of points. 
In general, we can represent any scheme $X$ as a functor defined on the (opposite) category of schemes $h_X : Y \mapsto {\rm Hom}_{\rm Sch_{/k}} (Y, X)$, but we will mostly focus on the affine case in this lecture.
\end{rmk}
%In general, given any scheme $S$, we define the functor 
%\end{rmk}

We encode deformation problems in this sort of functor of points approach of describing algebraic structures. 
As we saw in the last lecture, the overarching theme of deformation theory is to study moduli spaces of some mathematical structures. 
At first pass in this lecture, the types of moduli spaces we consider will be defined as functors of the form
\[
\sX : \CAlg_k \to \Set .
\]
Not every functor $\sX$ comes from an algebra (or even a scheme) through the Yoneda embedding as above.
So these are more general objects than one would consider in a first course in algebraic geometry. 

\begin{dfn}
A {\em classical moduli problem} (or {\em prestack}) is a functor $\sX : \CAlg_k \to \Set$. 
\end{dfn}

\begin{rmk}
A moduli space $\sX$ parameterizes families of structures in the sense that maps $f : S \to X$ define an ``$S$-family" of structures. 
If $\sX$ is a classical moduli problem, and $A$ is a commutative $k$-algebra, the set $\sX(A)$ thus has the following interpretation. 
From our discussion in the beginning of this lecture, $\sX(A)$ is equal to the set of maps ${\rm Spec}(A) \to \sX$. 
Thus, $\sX(A)$ corresponds to families of objects over ${\rm Spec}(A)$. 
\end{rmk}

It is unwieldy to study this huge class of moduli problems. 
We turn our attention to ``deformation problems" which means that we are only looking at the moduli of structures infinitesimally near a fixed object. 
Here is a geometric example to explain the type of behavior we are trying to codify. 

\begin{eg}\label{eg: fml}
Suppose $A$ is a commutative algebra and let $\phi : A \to k$ be a homomorphism of algebras.
Denote by $X = {\rm Spec}(A)$ the corresponding affine scheme.  
By our discussion above, this homomorphism $\theta$ is equivalent to choosing a closed point $x \in X$. 
Let $\fm= \ker (\phi) \subset A$ be the corresponding maximal ideal.
If $B$ is any commutative algebra, define the set
\[
X^{\Hat{\;\;}}_x (B) = \{f : A \to B \; | \phi = \psi \circ f \; \; , \;\; \forall\; \psi : B \to k \}
\]
The condition on the homomorphism $f : A \to B$ on the right hand side says that for {\em all} homomorphisms $\psi : B \to k$, the homomorphism $\phi$ factors as
\[
\begin{tikzcd}
A \ar[dr, "\phi"'] \ar[rr,"f"] & & B \ar[dl, "\psi"] \\ 
& k & 
\end{tikzcd} .
\]
In other words, this is the set of all maps of schemes ${\rm Spec}(f) : {\rm Spec}(B) \to {\rm Spec}(A)$ such that all points $y \in {\rm Spec}(B)$ are mapped to $x$. 

This construction defines a functor
\beqn\label{fmlcpl}
X^{\Hat{\;\;}}_x : \CAlg_k \to \Set 
\eeqn
that we call the ``formal completion of $X$ along $x$".
\end{eg}

\begin{ex}
Let $A = k [t]$ be the polynomial algebra in one variable and $\phi : k[t] \to k$ be the homomorphism that sends a polynomial to its constant term so that $\ker \phi = (t)$ is the unique maximal ideal.
Show that the formal completion of ${\rm Spec}(A)$ along $\phi$ is represented by the commutative algebra
\[
k[[t]] := \lim_{n}  k[t] / (t)^n .
\]
More generally, suppose $A$ is any commutative algebra and $\fm \subset A$ is a maximal ideal with corresponding homomorphism $\phi : A \to A / \fm = k$.
Show that the formal completion of ${\rm Spec}(A)$ along $\varphi$ is represented by the commutative algebra
\[
A^{\Hat{\;\;}}_\fm = \lim_{n} A / \fm^n .
\]
\end{ex}

The formal completion functor is completely determined on a certain subcategory of commutative $k$-algebras, called {\em local Artinian algebras}.
One should think of Artinian algebras as being the precise mathematical language to speak of ``infinitesimal neighborhoods" of points inside of some big algebro-geometric object. 

\begin{dfn}
A $k$-algebra $A$ is {\em local} if it has a unique maximal ideal $\fm_A \subset A$. 
A local algebra is {\em Artinian} if it is finite dimensional as a $k$-vector space.
Let $\Art_k$ denote the category of local Artinian $k$-algebras.
\end{dfn}

\begin{rmk}
Whenever we say ``Artinian" in these notes we mean ``local Artinian".
We will often write the unique morphism $\psi_A : \to A / \fm_A = k$ and the corresponding closed point as $\star \in \Spec(A)$. 
\end{rmk}

The discussion in Example \ref{eg: fml} shows that $X^{\Hat{\;\;}}_x$ from (\ref{fmlcpl}) is equivalent to the data of the functor
\[
\begin{array}{cccc}
X^{\Hat{\;\;}}_x : & \Art_k & \to & \Set \\
& (B,\fm_B) & \mapsto & \{f : A \to B \; | \; \phi = \psi_B \circ f\} . 
\end{array}
\]
Here, we use the notation $\psi_B : B \to B / \fm_B \cong k$ for the point corresponding to the maximal ideal of $B$. 
In other words, the formal completion is completely determined by its value on Artinian algebras. 
In this way, Artinian algebras probe infinitesimal behavior near a fixed point. 

\begin{dfn}
A {\em formal pre-deformation problem} is a functor
\[
F : \Art_k \to \Set
\]
such that $F(k) = \{\star\}$, the set with one element. 
A map of formal pre-deformation problems is a natural transformation of functors. 
\end{dfn}

For an Artinian algebra $A$ and a formal pre-deformation problem $F$, we think of $F(A)$ as the set of classes of ``$A$-deformations" of the object $F(k) = \star$. 
Note that for each $A$ we have inside of $F(A)$ the trivial deformation $\star = F(k) \to F(A)$ provided by the unit $k \to A$. 

\begin{rmk}
If $\sX$ is a classical moduli problem, we have seen that $\sX(A)$ corresponds to families of objects over ${\rm Spec}(A)$. 
We can also think of a formal pre-deformation problem $F$ as parametrizing some (formal) moduli space $\Hat{\sX}$.
Since $F(k) = \star$ is a single point, there is a unique map ${\rm Spec}(k) \to \Hat{\sX}$. 
Moreover, the set $F(A)$ corresponds to families of objects $Z$ over ${\rm Spec}(A)$ that reduce to the unique object $Z_0$ under the natural map ${\rm Spec}(k) \to {\rm Spec}(A)$:
\[
\begin{tikzcd}
Z_0 \ar[r] \ar[d] & Z \ar[d] \\
{\rm Spec}(k) \ar[r] & {\rm Spec}(A) .
\end{tikzcd}
\]
\end{rmk}

\begin{eg}
We have just seen that every scheme $X$ together with a point $x \in X$ defines a formal pre-deformation problem $X^{\Hat{\;\;}}_x$. 
This construction works more generally for any classical moduli problem (or prestack). 
Indeed, if $\sX : \CAlg_k \to \Set$ is any classical moduli problem, and $x \in \sX(k)$, then we define the the formal pre-deformation functor
\[
\begin{array}{cccc}
\sX^{\Hat{\;\;}}_x : & \Art_k & \to & \Set \\
& (B,\fm_B) & \mapsto & \{y \in \sX(B) \; | \; \sX(\psi_B)(y) = x\} . 
\end{array}
\]
More categorically, we are defining $\sX^{\Hat{\;\;}}_x(B)$ as the fiber, over $x \in \sX(k)$, of the map $\sX(\psi_B) : \sX(B) \to \sX(k)$. 
In other words, $\sX^{\Hat{\;\;}}_x(B)$ fits into a pull-back square of sets
\[
\begin{tikzcd}
\sX^{\Hat{\;\;}}_x(B) \ar[d] \ar[r] & \sX(B) \ar[d, "\sX(\psi_B)"] \\
\{x\} \ar[r] & \sX(k) .
\end{tikzcd}
\]
\end{eg}

\section{Formal deformation problems}

\begin{dfn}\label{dfn: def}
A {\em formal deformation problem} is a formal pre-deformation problem $F : \Art_k \to \Set$ such that for every fiber product
\[
\begin{tikzcd}
D \ar[d] \ar[r] & B \ar[d, "\sigma"] \\
C \ar[r] & A 
\end{tikzcd}
\]
in $\Art_k$ the induced map
\[
F(D) \to F(B) \times_{F(A)} F(C)
\]
is
\begin{enumerate}
\item[(i)] surjective if $\sigma : B \to A$ is surjective;
\item[(ii)] bijective if $A = k$.
\end{enumerate}
\end{dfn}

\begin{rmk}
The category of Aritnian algebras is closed under pull-backs. 
Suppose that $A,B,C \in \Art_k$ and $C \to A$, $B \to A$ are homomorphisms.
Then, $D = B \times_A C$ is an Artinian algebra. 
As algebras, $D \subset B \times C$. 
The unique maximal ideal of $D$ is $\pi_B^{-1}(\fm_B) = \pi_C^{-1}(\fm_C)$ where $\pi_B : B \times C \to B$, $\pi_C : B \times C \to C$ are the projections. 
\end{rmk}

\begin{dfn}
A morphism $\sigma : B \to A$ is {\em small} if $\sigma$ is a surjection and $\ker(\sigma) \cdot \fm_B = 0$. 
We call a short exact sequence of Artinian algebras
\[
M \to B \xto{\sigma} A
\]
a {\em small extension} if $\sigma$ is small. 
Further, if $M$ is principal we call such an extension {\em elementary}. 
\end{dfn}

\begin{eg}
Consider the ring of dual numbers $k[\epsilon] / \epsilon^2$. 
The obvious map $k[\epsilon] / \epsilon^2 \to k$ is small (in fact elementary). 
More generally, for any $n$ the morphism $k[\epsilon]/\epsilon^{n+1} \to k[\epsilon]/\epsilon^n$ is small (in fact elementary). 
\end{eg}

\begin{ex} 
Show that every small homomorphism $B \to A$ can be written as a composition of elementary morphisms
\[
B \to A_1 \to A_2 \to \cdots \to A_N \to A .
\]
\end{ex}

\begin{rmk}
This example shows that Definition \ref{dfn: def} of a formal deformation problem is equivalent to requiring that the predeformation problem $F$ satisfy the following. 
If $\sigma : B \to A$ is a {\em small} morphism in $\Art_k$ and $C \to A$ is any morphism then the induced map
\[
F(B \times_A C) \to F(B) \times_{F(A)} F(C)
\]
is
\begin{enumerate}
\item[(i)] surjective if $\sigma : B \to A$ is surjective;
\item[(ii)] bijective if $A = k$.
\end{enumerate}
That is, we only have to check compatibility of pull-backs for small morphisms. 
\end{rmk}

\begin{prop} Let $F$ be a formal deformation problem. 
Define $T_F = F(k[\epsilon]/\epsilon^2)$. 
Then, $T_F$ has the natural structure of a $k$-vector space. 
\end{prop}
\begin{proof}
\end{proof}

On one hand, the tangent space is equal the set of all first order deformations, which we have just shown carries the natural structure of a $k$-vector space. 
In addition, the tangent space controls the set of lifts of deformations along morphisms of Artinian algebras. 

\begin{prop}
Let $F$ be a formal deformation problem and suppose 
\[
M \to B \xto{\sigma} A
\]
is a small extension of Artinian algebras.
Then, there is a canonical transitive action of $T_F \tensor_k M$ on the fibers of the induced map
\[
F(\sigma) : F(B) \to F(A) .
\] 
which is functorial with respect to morphisms of small extensions. 
In particular, given $a \in F(A)$, the set of lifts $\Tilde{a} \in F(B)$ of $a$ is a torsor for the abelian group $T_F \tensor_k M$. 
\end{prop}

We have not addressed the issue of {\em existence} of deformations. 

\begin{dfn}\label{dfn: obs}
Let $F$ be a pre-deformation problem. 
An {\em obstruction theory} $(V, \theta)$ of $F$ is the data of:
\begin{enumerate}
\item[(1)] a vector space $V$;
\item[(2)] for every small extension
\beqn\label{ses1}
M \to B \to A
\eeqn
a map $\theta_{B} : F(A) \to V \tensor_k M$;
\end{enumerate}
such that:
\begin{enumerate}
\item[(i)] if $A = k$ in (\ref{ses1}), then $\theta_{B} (\star = F(k)) = 0$ for every small extension;
\item[(ii)] the maps $\theta_{B}$ are functorial for maps of small extensions. 
\end{enumerate}
\end{dfn}

The collection of obstructions theories for a fixed pre-deformation problem form a category in a natural way. 
Indeed, we say a morphism of obstruction theories $f : (V, \theta) \to (W, \psi)$ is a map of vector spaces $f : V \to W$ such that $\theta_B = (f \tensor 1_M) \circ \theta_B$ for every small extension as in (\ref{ses1}). 

This easy lemma shows that obstruction theories recognize lifting problems for deformation functors. 

\begin{lem}
Suppose $(V, \theta)$ is an obstruction theory for $F$ and suppose 
\[
M \to B \xto{\sigma} A 
\]
is a small extension. 
If $a \in F(A)$ lifts to an element $\Tilde{a} \in F(B)$, then $\theta_B(a) = 0$. 
\end{lem}
\begin{proof}
We can pull-back the small exact sequence along $\sigma$ to obtain the trivial small exact sequence
\[
M \to B \oplus M \to B .
\] 
Let $\theta_{B \oplus M} : F(B) \to V \tensor_k M$ be the corresponding obstruction map. 
By functoriality, we know that if $\Tilde{a} \in F(B)$ is any lift of $a \in F(A)$ then $\theta_{B \oplus M} (\Tilde{a}) = \theta_B(a)$. 
Thus, to prove the lemma, it suffices to assume that the small exact sequence is trivial. 
The trivial small exact sequence is pulled back from the small exact sequence
\[
M \to M \to k
\]
along the natural map $\psi_B : B \to k$. 
The statement then follows from condition (i) in Definition \ref{dfn: obs}
\end{proof}

\begin{dfn}
An obstruction theory is {\em complete} if for every small exact sequence (\ref{ses1}) that $\theta(a) = 0$ if and only if there exists a lift $\Tilde{a} \in F(B)$ of $a$.
\end{dfn}

\begin{thm}
Let $F$ be a formal deformation problem. 
Then, there exists a unique universal complete deformation theory $(O_F, \theta)$.
Here, universal means that it is initial in the category of complete deformation theories. 
\end{thm}

\section{Deformations of schemes}

\begin{dfn}
Suppose $X$ is a scheme over $k$. 
If $A$ is any augmented commutative $k$-algebra, the an {\em $A$-deformation} of $X$ is a pair $(X_A, \varphi)$ where $X_A$ is a scheme flat over $A$ and $\varphi : X \to X_A$ is a morphism of schemes which induces an isomorphism $X \xto{\simeq} X_A |_{{\rm Spec}(k)}$. 
In other words, $(X_A, \varphi)$ fit into a pull-back diagram of schemes
\[
\begin{tikzcd}
X \ar[d] \ar[r, "\varphi"] & X_A \ar[d] \\ 
{\rm Spec}(k) \ar[r] &{\rm Spec}(A) 
\end{tikzcd} 
\]
where the right vertical arrow is flat.
\end{dfn}

The {\em trivial} $A$-deformation of $X$ is given by $X_A = X \times \Spec(A)$ where $\varphi : X \to X_A$ maps $x \in X$ to $x \times \{\star\} \in X \times \Spec(A)$.
A {\em map} of $A$-deformations $f : X_A \to X_A'$ is a map of schemes over $\Spec(A)$ that restricts to the identity on $X$. 

\begin{rmk}\label{rmk: adef}
An $A$-deformation is equivalent to the data of a sheaf $\sF$ on $X$ of flat $A$-modules together with a map $\varphi^* : \sF \to \sO_X$ that induces an isomorphism of sheaves $\sF \tensor_A k \cong \sO_X$. 
In the definition above $\sF = \sO_{X_A}$ and $\varphi^*$ is the pull-back of the map of schemes $\varphi$. 
\end{rmk}

Given a scheme $X$, we can define the functor
\[
{\rm Def}_X : \Art_k \to \Set
\]
by sending an Artinian algebra $A$ to the the collection of isomorphism classes of $A$-deformations of $X$ with respect to the natural augmentation $A \to k$. 

\begin{prop}
For any $X$, the pre-deformation functor ${\rm Def}_X$ is a deformation functor. 
\end{prop}
\begin{proof}
Suppose $B \xto{\sigma_B} A \overset{\sigma_C}{\leftarrow} C$ are morphisms in $\Art_k$, where $\sigma_B$ is surjective. 
We want to show that the induced map
\[
\Psi : \Def_X (B \times_A C) \to \Def_X(B) \times_{\Def_X(A)} \Def_X(C)
\]
is surjective. 

Let's unpack this a bit. 
An element in the codomain of $\Psi$ is a pair $(X_B, \varphi_B) \times (X_C,\varphi_C)$ where:
\begin{itemize}
\item $X_B,X_C$ are flat over $\Spec(B), \Spec(C)$, respectively;
\item $\varphi_B : X \to X_B$, $\varphi_C : X \to X_C$ are maps;
\end{itemize}
such that 
\begin{itemize}
\item $\varphi_B,\varphi_C$ induce isomorphisms over $\Spec(k) \to \Spec(B), \Spec(C)$, respectively; 
\item there is an isomorphism of schemes
\[
f : X_B |_{\Spec(A)} \xto{\cong} X_C |_{\Spec(B)}
\] 
intertwining $\varphi_B,\varphi_C$. 
\end{itemize} 

The map $\Psi$ sends a deformation $(\Tilde{X}, \Tilde{\varphi})$ over $\Spec(B \times_A C)$ to its restrictions
\[
(\Tilde{X} |_{\Spec(B)}, \Tilde{\varphi}_B) \times (\Tilde{X} |_{\Spec(C)}, \Tilde{\varphi}_C) .
\]
where ...
We denote by $(Y, \sO_Y)$ either of the isomorphic schemes $(X_B, \sO_B) \cong_f (X_C, \sO_C)$. 
Note that there are natural maps $X_B \leftarrow Y \to X_C$. 
Define $Z$ to be the pushout in the category of ringed spaces
\[
\begin{tikzcd}
Y \ar[r] \ar[d] & X_B \ar[d] \\ X_C \ar[r] & Z .
\end{tikzcd}
\] 
The following is a technical lemma.
\begin{lem} The ringed space $(Z,\sO_Z)$ is a scheme.
\end{lem}

By assumption, the compositions $X \to Y \to X_B \to Z$ and $X \to Y \to X_C \to Z$ are equal and exhibit a map $\varphi : X \to Z$. 
Moreover, note that there are maps of schemes
\[
X_B \to \Spec(B) \to \Spec(B \times_A C)
\]
\[
X_C \to \Spec(C) \to \Spec(B \times_A C) .
\] 
By the universal property of pushouts, these maps provide us a unique map rendering the diagram commutative
\[
\begin{tikzcd}
Y \ar[r] \ar[d] & X_B \ar[d] \ar[ddr] & \\ 
X_C \ar[r] \ar[drr] & Z \ar[dr, dotted] & \\ 
& & \Spec(B \times_A C) .
\end{tikzcd}
\] 

\begin{lem}
The dotted map is flat.
\end{lem}

We have just constructed a pair $(Z, \varphi : X \to Z)$ where $Z$ is flat over $\Spec(B \times_A C)$. 
It is immediate to check that $\varphi$ induces an isomorphism $\varphi |_{\Spec(k)} : X \xto{\cong} Z|_{\Spec(k)}$ so that $(Z, \varphi)$ is an $B \times_A C$-deformation of $X$. 
Also, as an exercise one can check that $\Psi$ maps $(Z, \varphi)$ to the pair $(X_B, \varphi_B) \times (X_C,\varphi_C)$ we started with.

It remains to see that $\Psi$ is an isomorphism when $A = k$. 
\end{proof}

\begin{lem}
Suppose $X$ is an affine variety. 
Then every $A$-deformation is isomorphic to the trivial one.
\end{lem}

\begin{proof}
Let $X = \Spec(B)$.
%First, we prove the assertion for a first order deformation. 
%That is, the case $A = k[\epsilon] / \epsilon^2$. 
By Remark \ref{rmk: adef}, the data of an $A$-deformation is an algebra $B_A$ that is flat over $A$ together with a map of algebras $\varphi : B_A \to B$ inducing an isomorphism 
\beqn\label{def2}
\varphi : B_A \tensor_A k \xto{\cong} B.
\eeqn
In other words, we have a pushout diagram of commutative algebras
\[
\begin{tikzcd}
B & B_A \ar[l,"\varphi"']\\ 
k \ar[u] & A \ar[u] \ar[l] .
\end{tikzcd} 
\]

We first prove the assertion for the case that $A = k[\epsilon]/\epsilon^2$, so that $B_A$ is a first-order deformation. 
In this case, the map $\varphi$ induces the diagram of short exact sequences
\[
\begin{tikzcd}
& & & 0 & \\
0 \ar[r] & B \ar[r] & B_A \ar[dr, dotted, "\cong"'] \ar[r,"\varphi"] & B \ar[u] \ar[r] & 0 \\
& & & B \tensor_k k[\epsilon]/\epsilon^2 \ar[u] &  \\
& & & B \ar[u] & \\
& & & 0 \ar[u] & .
\end{tikzcd}
\]
The horizontal short exact sequence follows from the condition (\ref{def2}) that reads $B_A \tensor_{k[\epsilon]/\epsilon^2} k = B_A / \epsilon B_A \cong B$ and the fact that $\epsilon B_A = B$. 
The vertical short exact sequence comes from the trivial $k[\epsilon]/\epsilon^2$-deformation and is a sequence of $k[\epsilon]/\epsilon^2$-modules
By flatness, the vertical short exact sequence is split, and hence the map $\varphi$ lifts as shown. 
It is an isomorphism by the horizontal short exact sequence. 

%The trivial $A$-deformation of $B$ leads to the short exact sequence
%\[
%0 \to A \to B \tensor_k A \to B \to 0 .
%\]
%This exhibits $B \tensor_k A$ as an $A$-extension, hence by flatness is trivial as an $A$-module. 
%Thus, the map $\varphi : B_A \to B$ lifts to a map of $A$-algebras $\varphi : B_A \to B \tensor_k A$, which is an isomorphism since $B_A$ is an $A$-deformation. 

\end{proof} 

For a general scheme, it is difficult to completely characterize the tangent and obstruction spaces. 
In the case that $X$ is a smooth variety, however, we have the following clean geometric result.

\begin{thm}
Suppose $X$ is a smooth variety over $k$. 
Then:
\begin{itemize}
\item The tangent space $T_{{\rm Def}_X}$ is isomorphic to $H^1(X, T_X)$. 
\item The universal obstruction space $O_{{\rm Def}_X}$ is isomorphic to $H^2(X, T_X)$. 
\end{itemize}
In particular, if $H^2(X, T_X) = 0$ (say, if $X$ is a Riemann surface) then there always exists deformations of $X$. 
\end{thm}


\brian{Fact: affine schemes have no deformations}

\end{document}