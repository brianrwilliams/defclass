\documentclass[11pt]{amsart}

\usepackage{macros}

\linespread{1.25}

\usepackage[final]{pdfpages}

\setcounter{tocdepth}{2}

\title{Lecture 6: Model categories, I}

\def\Fun{{\sf Fun}}
\def\colim{{\rm colim}}
\def\Set{{\sf Set}}
\def\Top{{\sf Top}}

\begin{document}
\maketitle

\section{Some categorical remarks}

\subsection{(Co)Limits}

We recall some general notions in ordinary category theory. 
For a textbook reference see \cite{MacLane}.

Let $\sC$ and $\sI$ be categories. 
Unless otherwise stated, the category $\sI$ is assumed to be small.
For each object $X \in \sC$ let $\ul{X} : \sI \to \sC$ be the functor that sends every object of $\sI$ to $X \in \sC$ and every morphism to the identity. 
This construction extends to a functor
\[
\ul{(-)} : \sC \to \Fun(\sI, \sC) .
\]

\begin{dfn}
Suppose $\sI$ is a small category and let $F : \sI \to \sC$ be a functor. 
A (small) {\bf colimit} of $F$ is an object $X \in \sC$ together with a natural transformation
\[
t : F \to \ul{X}
\]
that is {\em initial} among all such natural transformations. 
In more detail, one requires that for every $Y \in \sC$ and every natural transformation $s : F \to \ul{Y}$, there exists a unique map $s' : X \to Y$ making $\ul{s'} t = s$. 
\end{dfn}

Any two colimits are naturally isomorphic. 
If the colimit of a functor $F : \sI \to \sC$ exists, we will write it as $\colim_{\sI} F \in \sC$. 

Note that for any functor $G : \sC \to \sD$ there is a natural map
\[
\colim_{\sI} (G \circ F) \to G \left(\colim_{\sI} F\right)
\]
We say that the functor $G : \sC \to \sD$ {\em preserves} the colimit defined by $F : \sI \to \sC$ if this map is an isomorphism.

Many familiar categories posses the property that {\em all} (small) colimits exist. 
These include $\Set, \Top$, and $\Vect_k$.
Also colimits are functorial in the natural way. 

\subsubsection{Examples of colimits}

\begin{eg}
\begin{enumerate}
\item[(1)] Suppose $\sI$ is the empty category and $F$ is the unique functor. 
If the colimit of $F$ exists, it is the initial object for the category $\sC$. 
\item[(2)] 
Let $\sI = \{0,1\}$ be the category with two objects and no non-identity morphisms. 
Then, $\colim_{\sI} F$, if it exists, is denoted $F(0) \sqcup F(1)$ and called the {\em coproduct} of $F(0)$ and $F(1)$. 
\item[(2)] Let $\sI = \{b \leftarrow a \to c\}$.
If $\colim_{\sI} F$ exists, it is called a {\em push-out} of the diagram $F(b) \leftarrow F(a) \to F(c)$. \\

\item[(3)]

\begin{dfn}
A category $\sI$ is called {\bf filtered} if, for any finite category $\sJ$ and functor $J : \sJ \to \sI$, there exists an object $i \in \sI$ and a natural transformation $F \to \ul{i}$. 
\end{dfn}

If $F : \sI \to \sC$ is a functor, $\sI$ is filtered, and $\colim_{\sI} F$ exists, then the colimit is called a {\em filtered colimit}. 

\item[(4)] A natural example of a filtered category is the poset
\[
\sI = \ZZ_+ = \{0 \to 1 \to 2 \to \cdots \} .
\]
Resulting colimits are special types of filtered colimits called {\em sequential colimits}. 
\end{enumerate}
\end{eg}

\subsubsection{}
The notion of a limit is defined in a dual way. 
The aforementioned categories also admit all limits. 

\section{The idea of a model category}
Roughly, the theory of model categories was developed to better handle the notion of a ``homotopy equivalence". 
For us, the fundamental example of a homotopy equivalence is a quasi-isomorphism of dg vector spaces. 

The first, and perhaps most obvious, attempt to encode homotopy, or {\em weak}, equivalences in a category is to prescribe some class of morphisms that are well-behaved with respect to composition. 
The definition is the following. 

\begin{dfn}
A {\bf category with weak equivalences} is a category $\sC$ together with a set
\[
\sW \subset {\rm Mor}(\sC)
\]
such that
\begin{enumerate}
\item If $f$ is an isomorphism, then $f \in \sW$;
\item if $f,g$ are morphisms such that $f \circ g$ exists then: if two of $f, g, f \circ g$ are in $\sW$ then the third is as well (`two out of three").
\end{enumerate}
\end{dfn}

\subsection{Model categories}

\begin{dfn}
Let $\sC$ be a category and $K \subset {\rm Mor}(\sC)$ be a subset of morphisms. 
A morphism $f : X \to Y$ in $\sC$ has the {\bf left lifting property} (LLP) with respect to $K$ if for any morphism $g : W \to Z$ in $K$ and solid line diagram
\[
\begin{tikzcd}
X \ar[r] \ar[d,"f"] & W \ar[d,"g"] \\ 
Y \ar[r] \ar[ur, dotted, "\exists h"'] & Z 
\end{tikzcd}
\]
there exists a dotted map $h : Y \to W$ making the total diagram commute. 
Dually, we say $f : X \to Y$ has the {\bf right lifting property} (RLP) with respect to $K$ if
for any morphism $g : W \to Z$ in $K$ and solid line diagram
\[
\begin{tikzcd}
W \ar[r] \ar[d,"g"] & X \ar[d,"f"] \\ 
Z \ar[r] \ar[ur, dotted, "\exists h"'] & Y
\end{tikzcd}
\]
there exists a dotted map $h : Z \to X$ making the total diagram commute. 
\end{dfn}

\subsection{Retracts}

Let $\sC$ be any category.
An object $X \in \sC$ is said to be a retract of $Y \in \sC$ if there exists morphisms $i : X \to Y$ and $r : Y \to X$ such that $r \circ i = \id_X$. 

\begin{eg}
A dg vector space $V$ is a retract of $W$ if and only if there exists a dg vector space $W$ such that $W \cong V \oplus Z$.
In this case, the maps can be taken to be the inclusion $i : V \hookrightarrow V \oplus Z$ and the projection $r : V \oplus Z \to V$. 
\end{eg}

\def\Arr{{\sf Arr}}

Given any category $\sC$ we can define its associated {\em arrow category} $\Arr(\sC)$ as follows. 
The set of objects is the set of morphisms ${\rm Mor}(\sC)$ of the starting category. 
A morphism from $f : X \to Y$ to $g : X' \to Y'$ is a commutative square
\[
\begin{tikzcd}
X \ar[r] \ar[d,"f"] & X' \ar[d, "g"] \\
Y \ar[r] & Y' .
\end{tikzcd}
\]

\begin{dfn}
A morphism $f \in {\rm Mor}(\sC)$ is a {\bf retract of a morphism} $g \in {\rm Mor}(\sC)$ if $f$ is a retract of $g$ in the arrow category $\Arr(\sC)$. 
Concretely, this means that we have a commutative diagram
\[
\begin{tikzcd}
X \ar[r] \ar[d,"f"] & X' \ar[d, "g"] \ar[r] & X\ar[d, "f"] \\
Y \ar[r] & Y' \ar[r] & Y 
\end{tikzcd}
\]
such that both horizontal compositions are the identity. 
\end{dfn}

Retracts are well-behaved under many categorical properties.
For instance, the following is easy to check.

\begin{lem}
If $g$ is an isomorphism and $f$ is a retract of $g$, then $f$ is an isomorphism. 
\end{lem}

We will also consider the behavior of retracts with respect to the lifting properties we just defined in the previous section.  
To get a flavor of he type of arguments one makes, consider the following easy lemma.
 
\begin{prop}
Suppose we have a commuting triangle
\[
\begin{tikzcd}
X \ar[rr,"f"] \ar[dr,"i"] & & Y \\
& Z \ar[ur, "p"] & .
\end{tikzcd}
\]
Then:
\begin{enumerate}
\item[(1)] 
if $f$ has LLP with respect to $p$, then $f$ is a retract of $i$;
\item[(2)] if $f$ has RLP with respect to $i$, then $f$ is a retract of $p$. 
\end{enumerate}
\end{prop}
\begin{proof}
For (1) consider the diagram
\[
\begin{tikzcd}
X \ar[r,"i"] \ar[d,"f"] & Z \ar[d,"p"] \\ 
Y \ar[r,equals] \ar[ur, dotted, "\exists h"'] & Y .
\end{tikzcd}
\]
The lift $h$ exists by the LLP. 
Such an $h$ defines the desired retract:
\[
\begin{tikzcd}
X \ar[r,equals] \ar[d,"f"] & X \ar[d, "i"] \ar[r,equals] & X \ar[d, "f"] \\
Y \ar[r,"h"] & Z \ar[r,"p"] & Y 
\end{tikzcd}
\]
\end{proof}

\subsection{The definition of a model category}

\begin{dfn}
A model category is a category with weak equivalences $(\sC, \sW)$ together with distinguished classes ${\rm Cof}, {\rm Fib} \subset {\rm Mor}(\sC)$ (cofibrations, fibrations respectively) satisfying the following axioms:
\begin{enumerate}
\item[(1)] for a map $f \in {\rm Mor}(\sC)$ one has:
\begin{itemize}
\item $f \in {\rm Cof} \iff f$ has the LLP with respect to ${\rm Fib} \cap \sW$;
\item $f \in {\rm Fib} \cap \sW \iff f$ has the RLP with respect to ${\rm Cof}$; 
\item $f \in {\rm Fib} \iff f$ has the RLP with respect to ${\rm Cof} \cap \sW$;
\item $f \in {\rm Cof} \cap \sW \iff f$ has the LLP with respect to ${\rm Fib}$. 
\end{itemize}
\item[(2)] One can factor any map $f : X \to Y$ in two ways:
\[
\begin{tikzcd}
X \ar[rr,"f"] \ar[dr, "{\rm Cof}"'] & & Y \\
& Z \ar[ur, "{\rm Fib} \cap \sW"'] & .
\end{tikzcd}
\]
and
\[
\begin{tikzcd}
X \ar[rr,"f"] \ar[dr, "{\rm Cof} \cap \sW"'] & & Y \\
& W \ar[ur, "{\rm Fib}"'] & .
\end{tikzcd}
\]
\item[(3)] The category $\sC$ has all (small) colimits and limits. 
\end{enumerate}
\end{dfn}

\begin{rmk}
We will refer to maps in ${\rm Cof} \cap \sW$ and ${\rm Fib} \cap \sW$ as acyclic cofibrations and acyclic fibrations respectively. 
\end{rmk}

\begin{eg}
The original example of a model category is the Quillen-Serre model structure on the category of topological spaces. 
The weak equivalences are given by weak homotopy equivalences.
These are maps of topological spaces that induce isomorphisms at the level of the homotopy groups. 
The cofibrations are those maps that arise as retracts of ``cell-gluings".
A cell-gluing $X \to Y$ is a map arising from pushout diagram of the form
\[
\begin{tikzcd}
S^n \ar[r,"\phi"] \ar[d,"\partial"'] & X \ar[d] \\
D^{n+1} \ar[r] & Y  .
\end{tikzcd}
\] 
The fibrations are the {\em Serre fibrations}.
These are maps that have the RLP with respect to maps of the form $(\id, 0) : D^n \hookrightarrow D^n \times I$. 
Check that $(\id,0)$ is indeed an acyclic cofibration!
\end{eg}

\section{The idea of localization}

Roughly, a localization of a category with weak equivalences is another category in which all of the weak equivalences are isomorphisms. 
Many such categories might exist, and we look at the {\em initial one} with respect to this property. 

\begin{dfn}
Let $(\sC, \sW)$ be a category with weak equivalences. 
A {\bf localization} of $\sC$ with respect to $\sW$ is a category $\sC[\sW^{-1}]$ equipped with a functor
\[
\Gamma : \sC \to \sC[\sW^{-1}]
\]
such that:
\begin{enumerate}
\item if $f \in \sW$ then $\Gamma(f) \in {\rm Mor}(\sC[\sW^{-1}])$ is an isomorphism;
\item if $F : \sC \to \sD$ is any functor such that $f \in \sW$ implies $F(f)$ is an isomorphism, then there is a unique factorization 
\[
\begin{tikzcd}
\sC \ar[rr, "F"] \ar[dr, "\Gamma"] & & \sD \\
& \sC[\sW^{-1}] \ar[ur, dotted, "\exists !"] & .
\end{tikzcd}
\] 
\end{enumerate}
\end{dfn}
It is immediate to see that when a localization exists it is unique. 

The problem of existence is one issue. 
The other, more serious problem for us, is coming up with a model for the localization when it exists. 
This is the problem that {\em model} categories solve!

%\brian{Whitehead localization of cat with weak equiv. Cof/fib construction for model cats. Why its better behaved for model cats.}

%\section{Adjoints and extensions}
%
%%\brian{Quillen adjunction, left/right Kan extensions, examples, Quillen equivalence}
%
%\section{Examples}
%
%
%
%\section{dgVect, dgLie, etc. Top.}

\end{document}