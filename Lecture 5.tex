\documentclass[11pt]{amsart}

\usepackage{macros}

\linespread{1.25}

\usepackage[final]{pdfpages}

\setcounter{tocdepth}{2}

\title{Lecture 4: Modules for dg Lie algebras}

\def\dgVect{{\sf Vect}^{\sf dg}}
\def\dgLie{{\sf Lie}^{\sf dg}}
\def\dgAss{{\sf Ass}^{\sf dg}}
\def\dgCAlg{{\sf CAlg}^{\sf dg}}
\def\dgLienil{{\sf Lie}^{\sf dg, nil}}
\def\Moduli{{\sf Moduli}}
\def\dgMod{{\sf Mod}^{\sf dg}}

\begin{document}
\maketitle

The enveloping algebra of a Lie algebra recognizes quasi-isomorphisms of dg Lie algebras in the following way/

\begin{lem}
A map of dg Lie algebras $f : \fg \to \fh$ is a quasi-isomorphism if and only if the induced map
\[
U(f) : U(\fg) \to U(\fh)
\]
is a quasi-isomorphism of associative dg algebras. 
\end{lem}
\begin{proof}
We consider the spectral sequences induced by the natural filtrations on the enveloping algebras of $\fg$ and $\fh$. 
The map $U(f)$ clearly preserves the natural filtration on the enveloping algebra and so it induces a map of spectral sequences. 
In particular, it induces a map at the $E_1$ page, which is simply the associated graded algebras
\[
{\rm Gr} \; U(f) : \Sym^*(\fg) \to \Sym^*(\fh) .
\] 
The map of spectral sequence converges to the map $U(f) : U(\fg) \to U(\fh)$. 

Now, it is an easy exercise to show that $\fg \to \fh$ is a quasi-isomorphism if and only if it induces a quasi-isomorphism of free commutative dg algebras $\Sym^*(\fg) \to \Sym^*(\fh)$. 
Thus, in one direction, if $\fg \to \fh$ is a quasi-isomorphism, we obtain a quasi-isomorphism at the $E_1$-page, and the result follows. 

In the other direction, if $U(\fg) \to U(\fh)$ is a quasi-isomorphism, then taking associated gradeds we obtain a quasi-isomorphism $\Sym^*(\fg) \to \Sym^*(\fh)$ and hence $\fg \to \fh$ is a quasi-isomorphism by the above remark. 
\end{proof}

\section{Modules and (co)homology}

\subsection{(dg) Modules}

Recall that to every dg vector space $V$ we can associated an associative dg algebra $\End_k(V)$ of endomorphisms. 

\begin{dfn}
Let $\fg$ be a dg Lie algebra.
A {\bf dg $\fg$-module} is a dg vector space $M$ together with a map of dg Lie algebras
\[
\rho : \fg \to \End_k(M) .
\]
A map of dg modules is defined in the obvious way. 
Let $\dgMod_{\fg}$ denote the category of dg $\fg$-modules. 
\end{dfn}

\begin{rmk}
By the universal property of the enveloping algebra, we see that a dg module for $\fg$ is equivalent to a (left) dg module for the associative dg algebra $U(\fg)$. 
\end{rmk}

The category of modules is an abelian category in the obvious way. 
Also, there is the notion of tensor product. 

\begin{dfn}
Suppose $M,N$ are two dg $\fg$-modules.
Define the tensor product dg $\fg$-module $M \tensor N$ to be the tensor product of underlying dg vector spaces with $\fg$-module structure given by
\[
\rho_{M} \tensor 1 + 1 \tensor \rho_N : \fg \to \End_k(M \tensor_k N) = \End_k(M) \tensor_k \End_k(N) .
\]
\end{dfn}

\begin{rmk}
If we think of $M,N$ as dg $U\fg$-modules, then 
\[
M \tensor N = M \tensor_{U\fg} N
\]
as $U \fg$-modules. 
\end{rmk}

\subsection{(Co)Homology}

As for ordinary modules, we have a pair of functors
\[
\begin{array}{cccl}
(-)^\fg : & \dgMod_\fg & \to & \dgVect_k  \\
& M & \mapsto & M^\fg = \{m \in M \; | \; x \cdot m = m \; , \; \forall x \in \fg\} \\
(-)_{\fg} : & \dgMod_{\fg} & \to & \dgVect_k \\
& M & \mapsto & M / \fg \cdot M 
\end{array}
\]
called the invariants/coinvariants respectively. 

\begin{rmk}
Note that 
\[
M^\fg = {\rm Hom}_{U \fg}(k, M)
\]
and 
\[
M_\fg = k \tensor_{U \fg} M .
\] 
\end{rmk}

\begin{lem}
Consider the functor
\[
{\rm triv}_\fg : \dgVect_k \to \dgMod_\fg
\]
that sends a dg vector space to the trivial $\fg$-module.
The functor $M \mapsto M^\fg$ is left adjoint to ${\rm triv}_\fg$. 
The functor $M \mapsto M_\fg$ is right adjoint to ${\rm triv}_\fg$. 
\end{lem}

As a consequence, taking invariants/coinvariants is left/right exact respectively. 
This motivates the following definition. 

\begin{dfn}
Let $\fg$ be a dg Lie algebra.
The {\bf Lie algebra homology} of $\fg$ is the left derived functor of coinvariants
\[
\begin{array}{cccc}
H_*(\fg ; -) : & \dgMod_\fg & \to & \Vect_k \\ 
& M & \mapsto & \LL_*(-)_\fg (M) \\ .
\end{array}
\]
Similarly, the {\bf Lie algebra cohomology} of $\fg$ is the right derived functor of invariants
\[
\begin{array}{cccc}
H^*(\fg ; -) : & \dgMod_\fg & \to & \Vect_k \\ 
& M & \mapsto & \RR_*(-)^\fg (M) \\ .
\end{array}
\]
\end{dfn}

\begin{rmk}
Using the Tor and Ext notation, we can write
\[
H^{\rm Lie}_*(\fg ; M) = {\rm Tor}^{U \fg}_* (k, M)
\]
and 
\[
H_{\rm Lie}^*(\fg ; M) = {\rm Ext}^*_{U \fg}(k, M) .
\] 
\end{rmk}

To compute the Lie algebra homology, for instance, one uses the usual trick for derived functors. 
By first finding a projective resolution, tensoring, then computing the cohomology. 
We proceed by finding a projective resolution of $U\fg$ 

\subsubsection{}

First, we sketch the following general construction for dg Lie algebras.
Given a dg Lie algebra $\fg$ define its {\em cone} to be the dg Lie algebra ${\rm Cone}(\fg)$ to be
\[
{\rm Cone}(\fg)_n = \fg_n \oplus \fg_{n-1}
\]
with differential
\[
\d_n = \left(\begin{array}{cc} \d_{\fg,n} & {\rm id}_{\fg_{n}} \\ 0 & \d_{\fg, n-1} \end{array}\right) : {\rm Cone}(\fg)_{n} = \fg_n \oplus \fg_{n-1} \to \fg_{n+1} \oplus \fg_n = {\rm Cone}(\fg)_{n+1}
\]
and bracket
\[
[(x,y), (x',y')] = \left([x,y]_{\fg}, [x,y'] + [y,x']\right) .
\]

\begin{lem}
There is a natural map of dg Lie algebras
\[
\fg \hookrightarrow {\rm Cone}(\fg) .
\]
Furthermore, ${\rm Cone}(\fg)$ is acyclic.
\end{lem} 

By functoriality of the enveloping functor, the associative algebra $U({\rm Cone}(\fg))$ is acyclic and hence a resolution for the trivial $U \fg$-module.
Thus, we have
\[
H^{\rm Lie}_*(\fg, M) = H^*\left(U({\rm Cone}(\fg)) \tensor_{U \fg} M \right) .
\] 

\begin{dfn}
The Chevalley-Eilenberg complex computing Lie algebra {\em homology} is the dg vector space
\[
\clieu_*(\fg ; M) := U({\rm Cone}(\fg)) \tensor_{U \fg} M .
\] 
Its cohomology is precisely the Lie algebra homology of $M$.
\end{dfn}

\begin{rmk}
Explicitly, as a graded vector space, the CE complex is of the form
\begin{align*}
\clieu_*(\fg ; M) & = U({\rm Cone}(\fg)) \tensor_{U \fg} M  \\ & = \left(\Sym(\fg[1]) \tensor_k U(\fg)\right) \tensor_{U \fg} M \\ &= \Sym(\fg[1]) \tensor_k M .
\end{align*}
Tracing through these isomorphisms, one can deduce that the differential is
\begin{align*}
\d_{CE} (x_1,\ldots, x_n) & = \sum_{i = 1}^n (\pm) x_1 \cdots x_{i-1} (\d x_i) x_{i+1} \cdots x_n \\ & + \sum_{i < j} (\pm) x_1\cdots \Hat{x_i} \cdots x_{j-1} [x_i,x_j] x_{j+1} \cdots x_n .
\end{align*}
\end{rmk}

\subsubsection{}

There is a completely analogous construction for Lie algebra cohomology. 

\begin{dfn}
The Chevalley-Eilenberg complex computing Lie algebra {\em cohomology} is the dg vector space
\[
\clie^*(\fg ; M) := U({\rm Cone}(\fg)) \tensor_{U \fg} M .
\] 
Its cohomology is precisely the Lie algebra homology of $M$
\[
H_{\rm Lie}^*(\fg ; M) = H^*\left(\Hom_{U \fg}(U({\rm Cone}(\fg)), M)\right) .
\]
\end{dfn}

\begin{rmk}
One can identify $\clie^*(\fg ; M)$ with a complex of the form
\[
\clie^*(\fg ; M) = \left(\Sym(\fg^\vee[-1]) \tensor_k M , \d^{CE}\right) .
\]
Note that, when $M = k$ there is an identification 
\[
\clie^*(\fg ; k) = \left(\clieu_*(\fg ; k)\right)^\vee = \Hom_k(\clieu_*(\fg ;k), k) .
\] 
\end{rmk}

\subsubsection{}

The CE complexes for homology and cohomology are functorial in the module input.
They both determine functors
\[
\begin{array}{cccc}
\clieu_*(\fg ; -) : & \dgMod_\fg & \to & \dgVect_k\\
\clie^*(\fg ; -) : & \dgMod_{\fg} & \to & \dgVect_k .
\end{array}
\]

We are interested in a different type of functoriality in the case that $M = k$, the trivial module. 
In this case, we write $\clieu_*(\fg) = \clieu(\fg;k)$ and similarly for cohomology.
A silly statement is that this trivial module is {\em universal} in the sense that it is a module for all dg Lie algebras.
Thus, we can contemplate the functoriality of homology/cohomology in the Lie algebra factor. 

\begin{lem}
The CE complex for homology/cohomology determine functors
\[
\begin{array}{cccc}
\clieu_*(-) : & \dgLie_k & \to & \dgVect_k\\
\clie^*(-) : & \dgLie_k & \to & \left(\dgVect_k\right)^{op} .
\end{array}
\]
Moreover, if $f : \fg \to \fh$ is a quasi-isomorphism of dg Lie algebras, then the induced maps
\[
\begin{array}{cccc}
\clieu_*(f) : & \clieu_*(\fg) & \to & \clieu_*(\fh)\\
\clie^*(\fg) : & \clie^*(\fh) & \to & \clie^*(\fg) 
\end{array}
\]
are quasi-isomorphisms. 
\end{lem}

\bibliographystyle{alpha}
%\bibliographystyle{spmpsci}  
\bibliography{def}


\end{document}