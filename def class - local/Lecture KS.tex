\documentclass[11pt]{amsart}

\usepackage{macros}

\linespread{1.25}

\usepackage[final]{pdfpages}

\setcounter{tocdepth}{2}

\title{Lecture ??: Deformations of complex structure}

\def\brian{\textcolor{blue}{BW: }\textcolor{blue}}

\def\Spec{{\rm Spec}}
\def\Def{{\rm Def}}

\begin{document}
\maketitle

\section{Macroscopic families}

\begin{dfn} 
A {\em holomorphic family of compact complex manifolds} is a proper holomorphic map
\[
\pi :  X \to B
\]
such that: $X, B$ are complex compact manifolds, $B$ is connected, and for each $x \in X$ the derivative $\pi_{*x} : T_x X \to T_{f(x)} B$ is surjective. 
In other words, $\pi$ is a proper holomorphic submersion. 
\end{dfn}

We think of $B$ parametrizing a family of complex manifolds $\{X_b = f^{-1}(b)\}_{b \in B}$. 

\begin{rmk}
There is a weaker notion of a ``smooth family" of complex compact manifolds. 
Here, one does not require that the base $B$ have a complex structure, and one considers smooth surjective submersions $\pi : X \to B$ such that the fibers $X_b \subset X$ have a complex structure. 
For more details see Chapter 4 of \cite{KSdef}. 
\end{rmk}

\begin{ex} 
The normal bundle of a submanifold $K \subset M$ is the orthogonal complement of $TK$ inside of $TM |_K$. 
In other words, it fits into a short exact sequence of bundles on $N$:
\[
0 \to TK \to TM |_K \to N_{K \subset M} \to 0 .
\]
Suppose that $\pi$ is a holomorphic family.
Show that for any $b$, $X_b = \pi^{-1}(b) \subset X$ is a closed submanifold and that the normal bundle $N_{X_b \subset X}$ is trivializable. 
\end{ex}

\begin{eg} {\em Elliptic curves}
\end{eg}

\begin{eg} {\em Hopf manifolds}.
Let $d > 1$ and fix complex numbers $\alpha_1,\ldots,\alpha_d$ such that $|\alpha_i| < 1$ for all $i\leq d$.
Consider the action of the infinite cyclic group $\ZZ$ on punctured affine space $\CC^d \setminus \{0\}$ generated by
\[
(z_1,\ldots, z_d) \mapsto (\alpha_1 z_1,\ldots, \alpha_d z_d) .
\]
This action is proper and discontinuous without fixed points. 
The quotient 
\[
X = \left(\CC^d \setminus 0\right) / \ZZ
\]
is called a {\em Hopf manifold}. 
It is a $d$-dimensional compact complex manifold that is diffeomorphic to $S^{2d-1} \times S^1$. 
\end{eg}

\section{Formal families of complex manifolds}

Recall that a local Artinian algebra is a commutative algebra with a unique maximal ideal that is finite dimensional as a $\CC$-vector space. 

\begin{dfn}
Fix a complex manifold $X$ and an Artinian algebra $A$.
A {\em formal $A$-deformation} (or simply just an $A$-deformation) of $X$ is a flat morphism
\[
\pi_A : X_A \to \Spec(A)
\]
together with a closed embedding $i : X \to X_A$ that induces a biholomorphism $i : X \xto{\simeq} \pi^{-1}(\star)$, where $\star = \Spec(\CC) \to \Spec(A)$ is the closed point. 
\end{dfn}

A formal deformation fits into a pull back diagram of locally ringed spaces
\[
\begin{tikzcd}
X \ar[d] \ar[r, "i"] & X_A \ar[d, "\pi"] \\ 
{\rm Spec}(\CC) \ar[r] &{\rm Spec}(A) 
\end{tikzcd} 
\]
and should look familiar to the general deformation picture of algebraic spaces. 
Just as in the algebraic case, for each complex manifold $X$ we can define the following functor
\[
\begin{array}{cccc}
\Def_X : & \Art_\CC & \to & \Set \\
& A & \mapsto & \left\{A{\rm -deformations\;of\;}X\right\} \;\; /\;\; \{{\rm iso}\}
\end{array}
\]
It is an exercise that uses very similar arguments as in the previous lectures to show that $\Def_X$ defines a deformation functor. 
We will utilize a more algebraic description of this functor.

\begin{lem}
The functor ${\rm Def}_X$ is a deformation functor. 
Moreover, it is equivalent to the functor 
\[
A \mapsto \left. \{{\rm sheaves\;} \sF{\rm \;of\;flat\;}A{\rm -algebras\;such\;that\;} \sF \tensor_{A} \CC \cong \sO_X\} \;\; \right/ \;\; \{{\rm iso}\} .
\]
\end{lem}

\subsection{The Kodaira-Spencer map}

By definition, a {\em first order} deformation of $X$ is an $A$-deformation where $A = \CC[\epsilon] / \epsilon^2$ is the ring of dual numbers. 
The collection of all first order deformations of $X$ is equal to the tangent space of $\Def_X$
\[
T_{\Def_X} = \Def_X(\CC[\epsilon]/\epsilon^2) .
\]
As we have seen in the previous lectures, it is a general fact that the tangent space is a $\CC$-vector space. 

\def\ks{{\rm ks}}

The Kodaira-Spencer map characterizes the tangent space of $\Def_X$, that is all first order deformations of $X$. 
It is of the form
\beqn\label{ks1}
\ks_X : T_{\Def_X} \to H^1(X, T_X). 
\eeqn
Its construction is the main goal of this subsection.  

To construct the Kodaira-Spencer map we will use a \v{C}ech description of the cohomology of the tangent sheaf $T_X$. 
A definition using a Dolbeault model will be given later. 

First, we have a lemma asserting that any $A$-deformation of a Stein manifold is trivial. 
The proof of this fact is completely analogous to the proof in the algebraic context where ``Stein" is replaced by affine.

\begin{lem}\label{lem: stein}
Any $A$-deformation of a Stein manifold is isomorphic to a trivial one. 
\end{lem}

Now, suppose a first order deformation $X_\epsilon \in T_{\Def_X} = \Def_X(\CC[\epsilon]/\epsilon^2$ is given. 
Pick a Stein open cover $\{U_\alpha\}$ of $X$ such that all double, $U_{\alpha\beta} = U_{\alpha} \cap U_\beta$, and triple, $U_{\alpha\beta\gamma} = U_\alpha \cap U_\beta \cap U_\gamma$, intersections are also Stein. 
By Lemma \ref{lem: stein}, we can pick isomorphisms of deformations
\[
\varphi_\alpha : U_\alpha \times \Spec(\CC[\epsilon]/\epsilon^2) \to X_\epsilon |_{U_\alpha} 
\]
for all $\alpha$. 
Moreover, for any $\alpha,\beta$ we obtain an automorphism of the trivial deformation
\[
\varphi_{\alpha\beta} \in {\rm Aut} \left(U_{\alpha\beta} \times \Spec(\CC[\epsilon]/\epsilon^2) \right) .
\]

\begin{lem}
Infinitesimally, on $U_{\alpha \beta}$, the automorphism $\varphi_{\alpha \beta}$ determines a vector field
\[
\theta_{\alpha \beta} \in \Gamma(U_{\alpha\beta}, T_X) .
\]
Moreover, the collection $\{\theta_{\alpha \beta}\}$ satisfy the \v{C}ech cocycle condition 
\[
\theta_{\alpha \beta} - \theta_{\alpha \gamma} + \theta_{\beta \gamma} = 0 .
\]
\end{lem}

By the lemma, the first order deformation $X_\epsilon$ determines an element $\{\theta_{\alpha \beta}\} \in C^1(\{U_{\alpha}\}, T_X)$. 
In cohomology, we obtain a class
\[
[ \{\theta_{\alpha \beta}\} ] = H^1(\{U_{\alpha}\}, T_X) \cong H^1(X, T_X) .
\]
The second isomorphism is a standard fact about \v{C}ech cohomology and Stein covers. 

\begin{thm}
There is a map of $\CC$-vector spaces 
\[
\begin{array}{cccc}
\ks_X : & \Def_X & \to & H^1(X, T_X) \\
 & X_{\epsilon} & \mapsto & [\{\theta_{\alpha \beta}] .
 \end{array}
 \]
Moreover, $\ks_X(X_{\epsilon}) = 0$ if and only if $X_\epsilon$ is isomorphic to the trivial deformation.
\end{thm}
\end{document}