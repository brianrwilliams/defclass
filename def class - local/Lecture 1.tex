\documentclass[11pt]{amsart}

\usepackage{macros}

\linespread{1.25}

\usepackage[final]{pdfpages}

\setcounter{tocdepth}{2}

\def\Spec{{\rm Spec}}

\title{Lecture 1: Introduction to the class}

\def\brian{\textcolor{blue}{BW: }\textcolor{blue}}

\begin{document}
\maketitle

\section{Moduli}

In this course, we are broadly interested in studying ``moduli spaces of structures".
This is the fundamental context of deformation theory. 
What is a ``moduli space?"
Depending on who you ask, you will get a plethora of answers, but one common theme is that moduli spaces describe mathematical structures that can be ``varied". 

\begin{evdfn}
A moduli space $\sM$ parametrizes a ``universal family of structures". 
Given a space $X$ and a map $f : X \to \sM$ is equivalent to prescribing an $X$-family of structures. 
\end{evdfn} 

\begin{eg}
In topology, one studies the space $B \GL(m)$. 
This is the classifying space of rank $m$ vector bundles in the sense that homotopy classes of maps $f : X \to B \GL(m)$ are equivalent to rank $m$ vector bundles on $X$ modulo isomorphism. 
Thus $B \GL(m)$ describes the ``moduli space of vector spaces of dimension $m$". 
\end{eg}

\begin{eg}
The moduli space of Riemann surfaces of genus $g$ is very familiar in algebraic geometry. 
It is defined by the property that the collection of maps 
\[
f : X \to \sM_g
\]
is equivalent to prescribing a (flat) family over $X$ of Riemann surfaces of genus $g$. 
\end{eg}

\subsection{A more precise notion}

Let $k$ be a field of characteristic zero and suppose $A$ is a commutative $k$-algebra.
Consider the functor
\[
\begin{array}{cccc}
h_{{\rm Spec}(A)} : & \CAlg_k & \to & \Set \\
& B & \mapsto & {\rm Hom}_{\CAlg_k}(A,B) 
\end{array}
\]
The assignment $A \mapsto h_{{\rm Spec}(A)}$ defines a functor
\[
h : \CAlg^{op}_k \to {\sf Fun}(\CAlg_k , \Set) .
\]
This is the Yoneda embedding for the category of commutative algebras. 

We encode deformation problems in this sort of functor of points approach of describing algebraic structures. 
Consider a functor of the form
\[
\sM : \CAlg_k \to \Set .
\]
Not every $\sX$ comes from an algebra (or even a scheme) through the Yoneda embedding as above.
So these are more general objects than one would consider in a first course in algebraic geometry. 

\begin{dfn}
A {\em classical moduli space} (or {\em prestack}) is a functor $\sM : \CAlg_k \to \Set$. 
\end{dfn}

A moduli space $\sM$ parameterizes families of structures in the sense that maps $f : S \to X$ define an ``$S$-family" of structures. 
If $\sM$ is a classical moduli problem, and $A$ is a commutative $k$-algebra, the set $\sM(A)$ thus has the following interpretation. 
From our discussion above, $\sM(A)$ is equal to the set of maps ${\rm Spec}(A) \to \sM$. 
Thus, $\sM(A)$ corresponds to families of objects over ${\rm Spec}(A)$. 

Often times it is very difficult to describe a general $S$-family of structures. 
There are two types of $S$ we will mostly restrict ourselves to in this lecture.

\subsubsection{}

Consider the ring $k[\epsilon]$ of polynomials in a single variable $\epsilon$. 
This ring has a truncation $k[\epsilon] / \epsilon^2$ of polynomials defined modulo $\epsilon^2$.
This is sometimes called the ring of ``dual numbers". 
As a $k$-vector space, $k[\epsilon] / \epsilon^2 = k \oplus k$.
Moreover, the unique maximal ideal $\epsilon \cdot k \subset k[\epsilon]/\epsilon^2$ defines an algebra homomorphism $k[\epsilon]/\epsilon^2 \to k$. 
Thus, the spectrum $\Spec(k[\epsilon]/\epsilon^2)$ is topologically a point $\star$. 

A $\Spec(k[\epsilon]/\epsilon^2)$-family of structures is defined by a map $f : \Spec(k[\epsilon]/\epsilon^2) \to \sM$.
Topologically, it is specified by where the unique closed point is sent $\star \mapsto X$.
Such a family is called the ``first-order" deformations of $X$. 
We will see many examples below. 

\begin{rmk}
In the next lecture, we will see how $\sM(k[\epsilon]/\epsilon^2)$ can be thought of as the tangent space of $\sM$ at $X$.
\end{rmk}

\subsubsection{}

The second type of family we will look at today is a ``formal family". 
This is very similar to the last example, where we replace the dual numbers $k[\epsilon]/\epsilon^2$ by the power series algebra $k[[\epsilon]]$.
Again, topologically, a $\Spec(k[[\epsilon]])$-family is specified by an object $X \in \sM$. 
We will call such families ``formal deformations" of $X$. 

\begin{rmk}
In the next lecture we will see how to carefully identify $\sM(k[[\epsilon]])$ with the formal neighborhood of the point $X$ in $\sM$. 
\end{rmk}

In even more generality we have the following.
For now, we will be vague about what a ``small" algebra $A$ is. 
In particular, it means that there is a unique point $\star$ in its spectrum $\Spec(A)$. 

\begin{vdfn} 
A ``formal moduli space" is an assignment $F$ of a set $F(A)$ to a ``small" commutative algebra $A$ satisfying some coherence conditions.
In particular, the algebras $k[\epsilon]/\epsilon^2$, $k[[\epsilon]]$ are ``small", so that it makes sense to talk about first-order and formal deformations of $F(\star)$.
\end{vdfn}

The main goal in this class is to give a complete {\em algebraic} characterization of formal moduli spaces. 

\section{Some examples}

\subsection{Deformations of algebras}

Let $k$ be a field and $A$ a finite-dimensional $k$-vector space. 
An associative multiplication is a bilinear map
\[
m : A \tensor_k A \to A
\]
satisfying 
\[
m(ab,c) = m(a,bc)
\]
for all $a,b,c \in A$. 
In other words, $m \circ (m \tensor {\rm id}_A) = m \circ ({\rm id}_A \tensor m)$ as maps $A^{\tensor 3} \to A$. 

A particular multiplcation $m$ is an element of the vector space ${\rm Hom}(A^{\tensor 2}, A)$, and the space of all multiplication cuts out some subspace. 
Given $A$, the space of {\em all} associative products can be describes as the fiber over zero of the (non-linear) map
\[
\begin{array}{ccccl}
\d_2 & : & {\rm Hom}(A^{\tensor 2}, A) & \to & {\rm Hom}(A^{\tensor 3}, A) \\
 & & \mu & \mapsto & \mu \circ (\mu \tensor {\rm id}_A) - \mu \circ ({\rm id}_A \tensor \mu)
\end{array}
\]
That is, 
\[
\{{\rm Associative\;algebra\;structures\;on\;}A\} = d_2^{-1}(0) .
\]

This is not yet the moduli space of associative products on $A$.
To obtain the moduli, we must identify associative products that are equivalent. 
This is the case if there exists a linear automorphism $\varphi : A \to A$ such that $\varphi \circ m = m' \circ \varphi$. 
Thus, $m$ is equivalent to $\varphi^{-1} \circ m \circ \varphi$. 

In conclusion, we find that the moduli space of all associative algebra structures on the $k$-vector space $A$ is 
\[
\d_2^{-1} (0) / \GL(A) .
\]

To obtain something meaningful, we must be careful when we take this quotient.
In any case, the description of this moduli space is abstract and is not very manageable for concrete calculations. 
If $\dim_k A = n$, the space ${\rm Hom}(A^{\tensor 2}, A)$ is of dimension $n^3$.
We are looking at an algebraic subvariety $\d_2^{-1}(0)$ cut out by a quadratic equation and then taking the quotient by a linear group that acts in a potentially wild way.

\subsubsection{}

Instead of trying to understand the full moduli space of associative products, we try to understand ``formal" deformations of a fixed associative product. 
Let $(A, m)$ be a given associative algebra. 
We consider associative products on the vector space $A[[\epsilon]] = A \tensor_k k[[\epsilon]]$ of the form
\[
m + \epsilon m^{(1)} + \epsilon^2 m^{(2)} + \cdots .
\] 
Note that modulo $\epsilon$ we get back the algebra we started with. 
This aligns with what we we called a ``formal deformation" above. 

For each $k$, we have $m^{(k)} \in {\rm Hom}(A^{\tensor 2}, A)$.
Of course, not every choice of such homomorphisms will result in an associative product.
We proceed by constructing the product ``order by order" in the formal parameter $\epsilon$.
Concretely, we view the power series algebra as a limit
\[
A[[\epsilon]] = \lim \left(\cdots \to A[\epsilon] / \epsilon^3 \to A[\epsilon] / \epsilon^2 \to A[\epsilon] / \epsilon^1 = A \right) .
\] 
Given a product defined modulo $\epsilon^{k+1}$
\[
m + \epsilon m^{(1)} + \cdots + \epsilon^k m^{(k)}
\]
we try to {\em lift} to a product defined modulo $\epsilon^{k+2}$
\[
m + \epsilon m^{(1)} + \cdots + \epsilon^k m^{(k)} + \epsilon^{k+1} m^{(k+1)} .
\]

\subsubsection{}

Even the first stage in this process is non-trivial. 
We are searching for $m^{(1)}$ so that $m + \epsilon m^{(1)}$ defines an associative product on $A[\epsilon]/\epsilon^2$. 
This is a ``first-order" deformation in the general language set up above.

The condition that $m + \epsilon m^{(1)}$ be associativity is equivalent to the vanishing of the expression
\begin{align*}
\d_2(m + \epsilon m^{(1)}) & = (m + \epsilon m^{(1)}) \circ \left((m + \epsilon m^{(1)}) \tensor \id_A - \id_A \tensor (m + \epsilon m^{(1)})\right) \\ & = \d_2(m) + \epsilon m^{(1)} \circ (m \tensor \id_A - \id \tensor m) \\ 
& \;\;\;\;\;\;\;\;\;\;\;\;\;\;\; + \epsilon m \circ (m^{(1)} \tensor \id_A - \id_A \tensor m^{(1)}) + O(\epsilon^2) \\ & = 
\epsilon \left(m^{(1)} \circ (m \tensor \id_A - \id \tensor m) + m \circ (m^{(1)} \tensor \id_A - \id_A \tensor m^{(1)})\right)
\end{align*}

We observe two features of this calculation. 
The lowest order $\epsilon^0$ term has dropped out since $m$ is itself an associative product, and is hence killed by $\d_2$. 
Secondly, the resulting expression is {\em linear} in the homomorphism $m^{(1)}$. 
We let 
\[
\begin{array}{ccccl}
\delta_2 & : & {\rm Hom}(A^{\tensor 2}, A) & \to & {\rm Hom}(A^{\tensor 3}, A) \\
 & & \mu & \mapsto & \mu \circ (m \tensor {\rm id}_A - \id_A \tensor m) - m \circ ({\rm id}_A \tensor \mu - \mu \tensor {\rm id}_A) .
\end{array}
\]
The map $\delta_2$ is linear and its kernel consists of those $m^{(1)}$ that define first-order deformations, by the above calculation.

When are two first-order deformations equivalent? 
This is the case when we have an algebra isomorphism
\[
\psi : \left(A[\epsilon]/\epsilon^2, m + \epsilon m^{(1)}\right) \xto{\cong} \left(A[\epsilon]/\epsilon^2, m + \epsilon m^{(1)'}\right)
\]
which is the identity modulo $\epsilon$.
Thus, the automorphism is of the form $\psi = \id_V + \epsilon \psi^{(1)}$.

\begin{ex}
Show that the condition that $\psi = \id_V + \epsilon \psi^{(1)}$ be an automorphism implies that $\psi^{(1)}$ satisfies
\[
[m, \psi^{(1)}] = m^{(1)} - m^{(1)'} .
\] 
Here, if $\alpha : A \to A$ is any linear map, we define 
\[
\begin{array}{cccl}
[m, \alpha] : & A \tensor A & \to & A \\
&  a \tensor b & \mapsto & m (\alpha(a), b) + m (a, \alpha(b)) - \alpha (m(a,b)) .
\end{array}
\]
Note that the map
\[
\delta_1 = [m, -] : {\rm Hom}(A,A) \to {\rm Hom}(A^{\tensor 2}, A)
\]
embeds linear maps as a subspace of maps $A^{\tensor 2} \to A$. 
\end{ex}
This exercise shows that $m^{(1)}$ and $m^{(1)} + [m, \psi^{(1)}]$ determine completely equivalent first-order deformations for any $\psi^{(1)} \in \Hom(A,A)$. 
Thus, the moduli space of first order deformations is of the form  
\[
\ker\left(\delta_2\right) \; / \; {\rm Im} \left(\delta_1\right) .
\]

\subsubsection{}

What about going further up the tower defining $A[[\epsilon]]$? 
For the next stage, give $m^{(1)}$ we want to find a product of the form
\[
m + \epsilon m^{(1)} + \epsilon^2 m^{(2)}
\]
defining a algebra structure on $A[\epsilon]/\epsilon^3$. 
In general, there is an {\em obstruction} to extending to a second-order deformation. 
This obstruction is measured by a certain quantity involving $m^{(1)}, m^{(2)}$.
More on this later.  

\subsection{Cochain complexes}

Suppose $(V, \d)$ is a $k$-linear cochain complex. 
This means that $V$ is a $\ZZ$-graded vector space $V = \oplus_{i \in \ZZ} V^i [-i]$, where the $k$-vector space $V^i$ is in degree $i$.
If $v \in V^i$ we say that $v$ has homogenous degree $i$ and write $|v| = i$. 
The differential is a linear map $\d : V \to V$ of degree one, so that $\d (V^i) \subset V^{i+1}$, satisfying $\d^2 = \d \circ \d = 0$. 

Given two cochain complexes $(V, \d_V)$ and $(W, \d_W)$ we have the vector space of {\em degree $n$ maps} from $V$ to $W$
\[
{\rm Hom}^n (V,W) = \{\varphi : V \to W \; | \; \varphi \; {\rm linear} \; , \; \varphi(V^i) \subset W^{i+n} \} .
\]
Define the $\ZZ$-graded vector space
\[
{\rm Hom}(V,W) = \bigoplus_{n \in \ZZ} {\rm Hom}^n(V,W)[-n] .
\] 
There is a natural differential on this graded vector space defined by
\[
\d_{\rm Hom} \varphi = \d_W \circ \varphi -  (-1)^{|\varphi|} \varphi \circ \d_V .
\]
A {\em cochain map} is $\varphi : V \to W$ such that $\d_{\rm Hom} \varphi = 0$. 

\subsubsection{}

What are the deformations of a cochain complex $(V,\d)$? 
The only piece of data we have to deform is the differential, so it is natural to expect that these come from deforming the differential.
As above, we look at formal deformations which have the form
\[
(V[[\epsilon]], \Tilde{\d}) = (V [[\epsilon]] , \d + \epsilon \d^{(1)} + \epsilon^2 \d^{(2)} + \cdots) 
\]
where $\epsilon$ is a formal parameter of grading degree zero. 

Of course, not every $\Tilde{\d}$ endows the graded vector space $V[[\epsilon]]$ with the structure of a cochain complex.
First of all, the differential is a linear endomorphism of degree $1$.
Thus, for each $k$ one has $\d^{(k)} \in \End^1(V)$.
The condition that $\Tilde{\d}$ define a differential is simply
\[
\Tilde{\d}^2 = 0 .
\] 
Just like the algebraic case, on the face of it this equation seems quite complicated. 
We proceed by constructing the deformation order by order in the parameter $\epsilon$. 

The first few terms are of the form:
\[
\cdots \to \left(V[\epsilon] / \epsilon^3, \d + \epsilon \d^{(1)} + \epsilon^2 \d^{(2)}\right) \to \left(V[\epsilon] / \epsilon^2 , \d + \epsilon \d^{(1)} \right) \to (V, \d) .
\] 

In order for $\d + \epsilon \d^{(1)}$ to define a first-order deformation we must solve the equation 
\[
0 = \left(\d + \epsilon \d^{(1)}\right)^2 = \epsilon \left(\d \circ \d^{(1)} - \d^{(1)} \circ \d\right)
\]
Note that the right hand side simply says that $\d_{\rm End} (\d^{(1)}) = 0$, where $\d_{\rm End}$ is the differential of the cochain complex $\End(V)$.
Thus, $\d + \epsilon \d^{(1)}$ is a first order deformation if and only if $\d^{(1)} \in \End^1(V)$ is $\d_{\rm End}$-closed. 

When are two first order deformations isomorphic?
Suppose $\d^{(1)}$ and $\d^{(1)'}$ are two first order deformations. 
They are isomorphic if there is an isomorphism of cochain complexes
\[
\psi : (V[\epsilon]/\epsilon^2 , \d + \epsilon \d^{(1)}) \xto{\cong} (V[\epsilon]/\epsilon^2, \d + \epsilon \d^{(1)'})
\]
such that $\psi \mod \epsilon = {\rm id}_V$. 
Thus, we can write $\psi = {\rm id}_V + \epsilon \psi^{(1)}$ 
Unravelling the condition that $\psi$ is a cochain map, we find that $\psi^{(1)}$ satisfies 
\[
\d_{\rm End} (\psi^{(1)}) = \d^{(1)} - \d^{(1)'} .
\] 
In other words, if $\d^{(1)}$ is any first-order deformation, and $\psi^{(1)} \in \End^0(V)$, then $\d^{(1)} + \d_{\rm End}(\psi^{(1)})$ determines an isomorphic first-order deformation. 

We conclude that the space of first-order deformations modulo equivalence is the cohomology
\[
H^1(\End(V)) = \ker \left(\d_{{\rm End}^1}\right) / {\rm Im} \left(\d_{{\rm End}^0} \right) .
\] 

%By the same calculation, see that the zeroeth cohomology
%\[
%H^0(\End(V)) 
%\] 
%has the following interpretation. 
%Given $\psi^{(1)} \in \End^0(V)$ such that $\d_{\End(V)} \psi^{(1)} = 0$ we obtain an automorphism $\id_V + \epsilon \psi^{(1)}$ of the trivial deformation $(V[\epsilon] / \epsilon^2, \d)$. 
%On the other hand, if $\eta \in \End^{(-1)}(V)$, then there is a commuting square
%\[
%\begin{tikzcd}
%(V[\epsilon] / \epsilon^2, \d) \ar[r, "

\subsubsection{}

Now, suppose that $\d^{(1)}$ is a given first-order deformation. 
We want to study the problem of lifting to a second order deformation of the form
\[
\left(V [\epsilon] / \epsilon^3 = \d + \epsilon \d^{(1)} + \epsilon^2 \d^{(2)}\right) .
\]
We must find $\d^{(2)}$ solving the equation
\[
0 = \left(\d + \epsilon \d^{(1)} + \epsilon^2 \d^{(2)}\right)^2 .
\] 
Since $\d^{(1)}$ is a given first-order deformation, we have $(\d + \epsilon \d^{(1)})^2 = 0 \mod \epsilon$. 
Thus, the equation above becomes
\[
0 = \epsilon^2 \left([\d, \d^{(2)}] + (\d^{(1)})^2\right) . 
\]
If this equation is {\em not} satisfied, then we do not obtain a deformation modulo $\epsilon^3$. 
In this sense, the right-hand side can be thought of as an {\em obstruction} to having a deformation. 

%\d_{\rm End} (\Tilde{\d}^{(2)}) + \frac{1}{2} [\Tilde{\d}^{(2)}, \Tilde{\d}^{(2)}] = 0 \mod \epsilon^2
%This equation is equivalent to
%\[
%
%\]
%where $\Tilde{\d}^{(2)} = \epsilon \d^{(1)} + \epsilon^2 \d^{(2)}$. 

\subsection{Flat bundles}

Let $M$ be a smooth manifold and $E \to M$ a vector bundle. 
A connection on $E$ is a linear map
\[
\nabla : \Gamma(E) \to \Gamma(T^*M \tensor E) = \Omega^1(M, E)
\] 
satisfying the Leibniz rule $\nabla (f s) = \d f \tensor s + f \nabla(s)$ for all $s \in \Gamma(E)$, $f \in C^\infty(M)$. 
We can extend $\nabla$ to a map (denoted by the same symbol)
\[
\nabla : \Omega^1(M, E) \to \Omega^2(M,E) = \Gamma(\wedge^2 T^* M \tensor E)
\]
by the rule $\nabla(\omega \tensor s) = (\d \omega) \tensor s - \omega \wedge \nabla(s)$. 
The composition $\nabla \circ \nabla : \Gamma(E) \to \Omega^2(M , E)$ is, in general, nonzero. 
It is, however, a $C^\infty(M)$-linear operator, and hence determines a section
\[
F_\nabla \in \Omega^2(M, \End(E))
\]
called the {\em curvature} of $\nabla$. 
A connection $\nabla$ is {\em flat} if $F_\nabla = 0$. 

We'd like to describe all flat connections of the fixed vector bundle $E$. 
We first unravel the flatness equation $F_\nabla = 0$. 
Any connection $\nabla$ can be written in the form
\[
\nabla = \d + \theta
\]
for $\theta \in \Omega^1(M, \End(E))$. 
Here, $\End(E)=E \tensor E^\vee$ is the endomorphism bundle. 
The condition that $F_\nabla = 0$ is equivalent to the {\em Maurer-Cartan equation}
\[
\d \theta + \frac{1}{2} [\theta, \theta] = 0 
\]  
where $[-,-]$ is defined using the commutator of endomorphisms together with the wedge product of forms. 

There is a notion of equivalence of connections through so-called ``gauge transformations". 
Given a section $g \in \Gamma({\rm Aut}(E))$, we define a new connection $g \cdot \nabla$ by
\[
(g \cdot \nabla)(s) = g^{-1} \nabla (g s) + (g^{-1} \d g) \tensor s .
\]
If we write $\nabla = \d + \theta$ and $g \cdot \nabla = \d + g \cdot \theta$, then 
\[
g \cdot \theta = g^{-1} \theta g + g^{-1} \d g .
\] 

One says that two connections $\nabla,\nabla'$ are {\em gauge equivalent} if there exists an automorphism $g$ such that $\nabla' = g \cdot \nabla$. 
Note that gauge equivalence preserves the space of flat connections. 

\begin{ex}
Suppose $\nabla,\nabla'$ are two gauge equivalent flat connections. 
Construct an isomorphism from the $\nabla$-flat sections 
\[
\Gamma^{flat}_{\nabla}(E) := \ker \nabla \subset \Gamma(E)
\]
to the $\nabla'$-flat sections $\Gamma^{flat}_{\nabla'}(E)$. 
\end{ex}

The moduli space of flat connections is the space of flat connections modulo gauge. 

\subsubsection{}
Suppose that $\theta$ defines a flat connection. 
Then, we see that to first order, $\theta + \epsilon \theta^{(1)}$ defines a flat connection if and only if 
\[
\d \theta^{(1)} + [\theta, \theta^{(1)}] = 0 .
\] 
One can check that two first-order deformations $\theta^{(1)}$ and $\theta^{(1)'}$ determine gauge equivalent connections if and only if there exists a section $X \in \Gamma(\End(E))$ such that
\[
\theta^{(1)} - \theta^{(1)'} = \d X + [\theta, X] .
\]
This shows that the moduli space of first-order deformations is precisely the first cohomology of the bundle $\End(E)$
\[
H^1(M, \End(E)) = H^1 \left(\Omega^*(M, \End(E)), \d + [\theta,-] \right)
\]

\section{A general structure}

So far, we have seen how various deformations problems can be turned into more familiar problems in cohomology or algebra. 
Common to all of the examples was a certain piece of structure called a {\em differential graded Lie algebra}. 

\subsection{Endomorphisms of a complex}

We begin with the example from Section \ref{sec: cochain}. 
Given a cochain complex $(V,\d)$ we saw that first order deformations were characterized by $H^1(\End(V))$. 
Moreover, there was an {\em obstruction} to finding a lift of a first order deformation which we found was an element in $H^2(\End(V))$. 

If $W$ is an ordinary vector space, the vector space $\End(W)$ has the structure of a {\em Lie algebra}. 
The Lie bracket is given by the commutator of endomorphisms. 
Similarly, the cochain complex $(\End(V), \d_{\End})$ has a sort of Lie bracket. 
To define it correctly, we need to be careful of signs:
\[
[\varphi, \psi] = \varphi \circ \psi - (-1)^{|\varphi| |\psi|} \psi \circ \varphi .
\] 
The bracket satisfies a graded version of skew symmetry and a graded version of the Jacobi identity. 
In addition, it is compatible with the differential in the sense that $\d_{\End}$ is a graded derivation for it. 
Such a triple $(\End(V), \d_{\End}, [-,-])$ is called a {\em differential graded Lie algebra}. 

This language offers a clean description of deformations of $(V,\d)$ in the following way. 
The space $H^1(\End(V))$ does not use the Lie bracket and still parametrizes first-order deformations. 
Now, the obstruction to lifting a first order deformation $\epsilon \d^{(1)}$ to a second order deformation $\Tilde{\d}^{(2)} = \epsilon \d^{(1)} + \epsilon^2 \d^{(2)}$ is of the form
\[
\d_{\rm End} (\Tilde{\d}^{(2)}) + \frac{1}{2} [\Tilde{\d}^{(2)}, \Tilde{\d}^{(2)}] = 0 \mod \epsilon^2 .
\]
This is called the {\em Maurer-Cartan equation} for the dg Lie algebra. 

We summarize the deformation theory of cochain complexes as:
\begin{itemize}
\item $H^0({\rm End}(V))$ is the Lie algebra of symmetries of a fixed deformation. 
\item $H^1(\End(V))$ is the space of first-order deformations modulo equivalence.
\item $H^2(\End(V))$ is the space where the obstruction to lifting first-order deformations lives.
\end{itemize}

\subsection{Hochschild cohomology}
We saw that the first-order deformations of an algebra $(A, m)$ can be realized as the quotient space $\ker(\delta_2) / {\rm im}(\delta_1)$. 
This expression should look familiar, as a cohomology. 
Indeed, we can think of space as the first cohomology of the complex
\[
\Hom(A, A) \xto{\delta_1} \Hom(A^{\tensor 2}, A) \xto{\delta_2} \Hom (A^{\tensor 3}, A) .
\] 
This is a piece of a well-studied object called the {\em Hochschild cohomology} of the algebra $A$. 
Indeed, this sequence can be extended to the left and right to yield
\[
\begin{array}{ccccccccccc}
& \ul{-1} & & \ul{0} & & \ul{1} & & \ul{2} & & \\
& A & \xto{[-,-]} & \Hom(A, A) & \xto{\delta_1} & \Hom(A^{\tensor 2}, A) & \xto{\delta_2} & \Hom (A^{\tensor 3}, A) & \to \cdots
\end{array}
\]
The first map is simply $a \mapsto [a,-]$. 
This sequence is actually a complex.
The Hochschild cohomology is the cohomology of this complex, denoted $HH^*(A, A)$. 

There is a dg Lie algebra structure on the (shift of the) Hochschild cohomology cochain complex. 
In degree zero, $\Hom(A,A) = \End(A)$ we know there is the obvious structure of a Lie algebra.
In the full complex, the differential is the one we've just described.
There is a natural extension of the Lie bracket in degree zero to one on the full complex.
It is perhaps easiest to think about pictorially. 

\begin{itemize}
\item $HH^0(A,A)$ is the center of $A$.
\item $HH^1(A,A)$ is equal to the space of derivations of $A$ modulo inner derivations. 
\item $HH^2(A,A)$ is equal to the space of first-order deformations of $A$ modulo equivalence.
\item $HH^3(A,A)$ is the space where the obstruction to extending a deformation lives.
\end{itemize}

\subsection{Flat bundles}

For the flat bundle case, there is also a dg Lie algebra floating around. 
For a fixed flat connection $\nabla = \d + \theta$ on a vector bundle $E \to M$, it is of the form
\[
\left(\Omega^{*}(M, \End(E)), \d + [\theta,-]\right) .
\]
The Lie bracket is induced from the obvious one on $\End(E)$ together with the wedge product of differential forms.
Notice that the Maurer-Cartan equation for this dg Lie algebra is the usual one from geometry. 

\subsection{The fundamental theorem of deformation theory}

The general heuristic, or ``experimental fact", widely believed well before its full proof appeared is the following. 

\begin{ef}
For every formal moduli space one can find a differential graded (dg) Lie algebra {\bf controlling} it. 
Equivalences of deformation problems are detected by quasi isomorphisms of dg Lie algebras.
\end{ef}

By controlling, we have been purposefully vague.
Part of what this means is that for any formal moduli space $F$, there is a dg Lie algebra $\fg_{F}$ whose first cohomology $H^1(\fg_F)$ labels the first-order deformations. 
Any deformation $\theta$ must satisfy the Maurer-Cartan equation in $\fg_{F}$:
\[
\d_{\fg_{F}} \theta + \frac{1}{2} [\theta,\theta] = 0 .
\] 
Making this heuristic precise is most of the problem. 
In any case, we can state the full result in slightly more categorical terms as:

\begin{thm}[Lurie, Hinich and Pridham]
There is an equivalence of ($\infty$-)categories between:
\begin{itemize}
\item the category of {\bf formal moduli spaces} over a field $k$ of characteristic zero and
\item the category of {\bf dg Lie algebras} over $k$. 
\end{itemize}
\end{thm} 

In this statement we have not yet made evident the exact mechanism that one can arrive at a dg Lie algebra from a formal moduli space, or vice-versa. 
Most of the work involved is in setting this up.
To formulate and prove this theorem we must use some technology in the theory of higher algebra and $\infty$-categories. 

\end{document}
 
