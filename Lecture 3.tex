\documentclass[11pt]{amsart}

\usepackage{macros}

\linespread{1.25}

\usepackage[final]{pdfpages}

\setcounter{tocdepth}{2}

\title{Lecture 3: dg Lie algebras and the Maurer-Cartan equation}

\def\brian{\textcolor{blue}{BW: }\textcolor{blue}}



\begin{document}
\maketitle

\section{Where are we going?}

This lecture marks the end of motivation from classical deformation theory and begins the bulk portion of course. 
Recall, we have defined the theory of formal moduli problems from a functor of points perspective. 
Precisely, a formal moduli problem over $k$ is a functor
\[
F : \Art_k \to \Set
\]
satisfying some hypotheses. 
In many examples, we found that such a formal moduli problem are ``controlled" by the cohomology of another algebraic object that was equipped with a Lie bracket of sorts. 

In this lecture we begin to formalize this algebraic notion, namely {\em dg Lie algebras}, as well as connect it back to deformation theory through the Maurer-Cartan equation. 
After introducing some definitions and foundational context, our goal in the next few lectures is to construct a functor
\[
\Def : \dgLie_k \to \Moduli_k
\]
from the category of dg Lie algebras to the category of formal moduli problems over $k$. 
Most of the rest of the course contemplates how far (or close) this functor is from being an equivalence.

\section{A introduction to dg Lie algebras}

%Associated to any smooth manifold $M$ is a natural vector bundle $TM \to M$ called its tangent bundle. 

Let $R$ be a commutative ring over $k$. 
A derivation of $R$ is a $k$-linear map $D : R \to R$ such that for all $a,b \in R$ one has
\[
D(ab) = D(a) b + a D(b) .
\] 
Let $\Der(R)$ be the vector space of all derivations.
Suppose $D_1,D_2$ are $R$-derivations and consider the composition $D_1 \circ D_2 : R \to R$. 
Applied to the product of two elements we compute
\[
(D_1 \circ D_2) (ab) = D_1(D_2(a) b + a D_2(b)) = (D_1 \circ D_2)(a) b + D_1(a) D_2(b) + D_2(a) D_1(b) + a (D_1 \circ D_2)(b) .
\] 
In particular, $D_1 \circ D_2$ is not a derivation. 
However, if we compute the flipped composition
\[
(D_2 \circ D_1) (ab) = (D_2 \circ D_1)(a) b + D_2(a) D_1(b) + D_1(a) D_2(b) + a (D_2 \circ D_1)(b)
\]
we notice that the {\em commutator}
\[
[D_1, D_2] := D_1 \circ D_2 - D_2 \circ D_1
\]
is a derivation. 
This equips the vector space $\Der(R)$ with the structure of a {\em Lie algebra}. 

\begin{dfn}
A {\bf Lie algebra} over $k$ is a vector space $V$ equipped with a bilinear map
\[
[-,-] : V \times V \to V
\]
called the {\em bracket} satisfying the conditions:
\begin{enumerate}
\item[(1)] Skew symmetry.
For all $x,y \in V$, one has $[x,y] = - [y,x]$. 
\item[(2)] Jacobi identity. 
For all $x,y,z \in V$ one has 
\[
[x,[y,z]] = [[x,y],z] + [y,[x,z]] .
\]
\end{enumerate}
A map of Lie algebras is a linear map $f : V \to W$ preserving the brackets. 
\end{dfn}

\begin{rmk}
Standard examples of Lie algebras from representation theory include the $n \times n$ matrix Lie algebras $\fgl(n), \fsl(n), \fu(n)$, etc.. 
More generally, if $W$ is any vector space we let $\fgl(W)$ be the Lie algebra of endomorphisms of $W$ with bracket given by the commutator. 
Note that for commutative rings $R$, there is an inclusion of Lie algebras $\Der (R) \hookrightarrow \fgl(R)$. 
\end{rmk}
 

\begin{rmk}
If $(A,\cdot)$ is any associative algebra, the the commutator $[a,b] := a \cdot b  - b \cdot a$ endows $A$ with the structure of a Lie algebra. 
Thus, there is a functor
\[
\Ass_k \to \Lie_k .
\]
\end{rmk}



\begin{eg}
One can associate a natural Lie algebra to any smooth (or complex) manifold $M$. 
Let $TM \to M$ be the tangent bundle and $T_M$ its space of smooth sections. 
The Lie bracket of vector fields 
\[
[-,-] : T_M \times T_M \to T_M
\]
endows $T_M$ with the structure of a Lie algebra.
Locally, $T_M$ is generated over $C^\infty(M)$ by symbols $\{\partial/ \partial x_1,\ldots,\partial / \partial x_{{\rm dim}(M)}\}$. 
The Lie bracket is defined locally by the formula
\[
\left[ f(x) \frac{\partial}{\partial x_i} , g(x) \frac{\partial}{\partial x_j} \right] = \frac{\partial f}{\partial x_j} (x) g(x) \frac{\partial}{\partial x_i} - f(x) \frac{\partial g}{\partial x_i} (x) \frac{\partial}{\partial x_j} 
\]
where $f,g \in C^\infty(M)$.
\end{eg}

\subsection{dg Lie algebras}

In this course a more general object than a plain Lie algebra will play a central role. 
We have already met the notion of a cochain complex as a $\ZZ$-graded vector space equipped with a {\em differential} that is square zero and of grading degree one. 
The notion of a dg Lie algebra marries this concept with that of an ordinary Lie algebra. 

\begin{dfn}
A {\bf dg Lie algebra} (dgla) over $k$ is:
\begin{enumerate}
\item[(i)] a $\ZZ$-graded $k$-vector space $V = \{V^n\}_{n \in \ZZ}$;
\item[(ii)] a degree $+1$ linear map $\d : V \to V$ called the {\em differential}.
This is equivalent to a collection of linear maps
\[
\d^n : V^n \to V^{n+1};
\]
\item[(iii)] a bilinear map of degree zero $[-,-] : V \times V \to V$ called the {\em bracket}. This is equivalent to a collection of maps
\[
[-,-] : V^m \times V^n \to V^{m+n}. 
\]
\end{enumerate}
This data must satisfy the conditions:
\begin{enumerate}
\item[(1)] Differential. $\d^2 = 0$ (Thus $(V,\d)$ is a cochain complex);
\item[(2)] Graded skew symmetry. For all $x,y \in V$ one has $[x,y] = (-1)^{|x||y|} [y,x]$;
\item[(3)] Graded Jacobi. For all $x,y,z \in V$ one has
\[
[x,[y,z]] = [[x,y],z] + (-1)^{|x||y|} [y,[x,z]];
\]
\item[(4)] Graded derivation.
For all $x,y\in V$ one has
\[
\d([x,y]) = [\d x, y] + (-1)^{|x|} [x, \d y] .
\]
\end{enumerate}
A map of dg Lie algebras is a linear map $f : V \to W$ of grading degree zero that preserves the differentials and Lie brackets.
We denote the resulting category by $\dgLie_k$. 
\end{dfn}

\begin{rmk} A dg Lie algebra is simply a Lie algebra object in the category of dg vector spaces (cochain complexes). 
\end{rmk}

\begin{rmk}
If $V$ is a dg Lie algebra then the bracket restricts to $V^0 \subset V$ and endows $V^0$ with the structure of an ordinary Lie algebra.
This defines a functor of ordinary Lie algebras into the category $\dgLie_k$. 

Since $\d^2 = 0$, it makes sense to consider the cohomology $H^*(V)$. 
The cohomology inherits the Lie bracket from $V$ and has the structure of a dg Lie algebra with differential identically zero. 
\end{rmk}

\begin{eg}
Let $(W, \d)$ be a dg vector space and consider its degree $n$ endomorphisms \footnote{If $W$ is a graded vector space we let $W[n]$ denote the graded vector space obtained by shifting all degrees down by $n$.
That is, $W[n]^i = W^{i+n}$. 
A linear endomorphism of degree $n$ on $W$ is then a degree zero map $\varphi : W \to W[n]$.}
\[
{\rm End}^n (V,W) = \{\varphi : W \to W \; | \; \varphi \; {\rm linear} \; , \; \varphi(W^i) \subset W^{i+n} \} .
\]
Define the $\ZZ$-graded vector space
\[
{\rm End}(W) = \bigoplus_{n \in \ZZ} {\rm Hom}^n(V,W)[-n] .
\] 
There is a natural differential on this graded vector space defined by
\[
\d_{\rm End} \varphi = \d_W \circ \varphi -  (-1)^{|\varphi|} \varphi \circ \d_V .
\]
Note that a {\em cochain map} is $\varphi : V \to W$ such that $\d_{\rm Hom} \varphi = 0$. 

The complex $(\End(W), \d_{\rm End})$ has a natural Lie bracket extending the ordinary one in degree zero. 
If $\varphi, \psi \in \End(W)$, define the graded commutator
\[
[\varphi, \psi] = \varphi \circ \psi - (-1)^{|\varphi| |\psi|} \psi \circ \varphi .
\] 
The bracket satisfies the graded version of skew symmetry and the graded version of the Jacobi identity (exercise). 
In addition, it is compatible with the differential in the sense that $\d_{\End}$ is a graded derivation for it. 
Thus, the triple $(\End(V), \d_{\End}, [-,-])$ is naturally a dg Lie algebra.
\end{eg}

\begin{eg}
For a slightly more general example, take any associative dg algebra $A$. 
Then, the graded commutator makes $A$ into a dg Lie algebra as well.
\end{eg}

\begin{ex}
Let $R$ be a commutative dg algebra. 
Define a derivation of degree $n$ to be a linear map $D : R \to R[n]$ satisfying
\[
D(ab) = D(a) b + (-1)^{n|a|} a D(b) .
\] 
Define the $\ZZ$-graded vector space $\Der(R) = \{\Der^n(R)\}_{n \in \ZZ}$ where $\Der^n(R)$ is the space of derivations of degree $n$.
Show that $\Der(R)$ has the natural structure of a dg Lie algebra in such a way that there is an inclusion of dg Lie algebras $\Der(R) \hookrightarrow \End(R)$.
\end{ex}

\section{Maurer-Cartan equation and nilpotency}

Throughout the rest of this section we let $(\fg, \d, [-,-])$ be a dg Lie algebra which we will simply refer to by $\fg$.
We have already noticed that the degree zero part of $\fg$ has the structure of an ordinary Lie algebra.
We are actually most interested in what goes on in higher cohomological degrees. 

\begin{dfn}
A degree one element $x \in \fg^1$ is a {\bf Maurer-Cartan element} of $\fg$ if it satisfies the equation
\beqn\label{MC}
\d x + \frac{1}{2} [x,x] = 0 .
\eeqn
Let $\MC (\fg) \subset \fg^1$ denote the set of Maurer-Cartan elements. 
\end{dfn}

Equation (\ref{MC}) is called the Maurer-Cartan equation.  
It cuts out a quadratic algebraic variety inside the linear space $\fg^1$. 

\begin{lem}\label{lem: easy}
Suppose $\alpha \in \MC(\fg)$ and define the new triple
\[
\fg_\alpha = (\fg, \d + [\alpha,-], [-,-]) .
\]
Then $\fg_{\alpha}$ has the structure of a dg Lie algebra. 
\end{lem}
\begin{proof}
We need to check that $\left(\d + [\alpha,-]\right)^2 = 0$. 
This equation is evidently equivalent to the Maurer-Cartan equation for $\alpha$. 
\end{proof}

\subsection{Gauge transformations}

Given any Lie group $G$, the tangent space at the unit $T_e G$ has the natural structure of a Lie algebra. 
Conversely, if $\fg$ is any Lie algebra there is a unique simply connected Lie group whose tangent space at the unit is $\fg$. 
This means that phenomena in Lie groups can be translated to problems in Lie algebras and vice-versa. 

For simplicity and motivation, we will momentarily work over $\CC$. 
Let $G$ be a Lie group with Lie algebra $\fg$. 
For any $X \in \fg$ there is a map of Lie algebras
\[
\tau_X : \CC \to \fg
\]
sending $t \mapsto t X$. 
Here, we view $\CC$ as a Lie algebra with trivial bracket. 
Via the correspondence between Lie groups and Lie algebras, there is then a unique map of Lie groups
\[
\gamma_X : \CC \to G .
\]
We are viewing $\CC$ as a Lie group with group operation given by addition. 
We define the {\em exponential map} by
\[
\exp : \fg \to G \;\; , \;\; X \mapsto \gamma_X(1) .
\] 

Given two elements $X,Y$ of $\fg$ we can consider the product $\exp(X) \exp(Y)$ in $G$.
A natural question is whether this product can be expressed in the form $\exp(Z)$ for some $Z \in \fg$. 
Since $\exp$ is a local diffeomorphism, this is possible whenever $X,Y$ are close enough to zero in $\fg$.
However, in general, $Z$ may not exist. 

\begin{eg}
Let $\fg = \fgl(2)$ and consider the elements
\[
X = \left(\begin{array}{cc} 0 & -1 \\ 1 & 0 \end{array} \right) \;\; , \;\; Y = \left(\begin{array}{cc} 0 & 1 \\ 0 & 0 \end{array} \right) .
\] 
Then, there is no matrix $Z$ such that $\exp(Z) = \exp(X) \exp(Y)$.
\end{eg}

When such a $Z$ exists, a combinatorial presentation for it in terms of $X,Y$ is given by the {\em Baker-Campbell-Hausdorff} formula.

\subsubsection{}

Now, let's go back to the case of a dg Lie algebra $(\fg, \d, [-,-])$. 
Notice that $\fg^0 \subset \fg$ is a sub dg Lie algebra. 
In fact, $\fg^n$ is a module for the Lie algebra $\fg^0$ through the adjoint action for each $n \in \ZZ$.  
In formulas, if $X \in \fg^0$ and $Y \in \fg^n$ then this action is
\[
\ad(X) (Y) = [X,Y]  \in \fg^n .
\]

The object we are interested in is the Maurer-Cartan set $\MC(\fg)$ of the dg Lie algebra. 
Although $\fg^0$ does not act on $\MC(\fg)$ in any natural way (in fact, $\MC(\fg)$ is rarely even a vector space) we can formally define the action of a group $\exp(\fg^0)$ on $\MC(\fg)$. 
For now, we think of $\exp(\fg^0)$ as the simply connected Lie group corresponding via Lie's theorem to $\fg^0$. 
In most examples this is subtle since $\fg^0$ may not be finite dimensional. 
Even so, we can make sense of the construction. 

Assuming that this exponential exists, we observe the following. 
Since $\fg^1$ is a $\fg^0$-module, we see that $\fg^1$ is formally a $\exp(\fg^0)$-representation. 
By formal properties of the exponential, we see that if an element $\alpha \in \fg^1$ solves the Maurer-Cartan equation then so does $\exp(X) \cdot \alpha$ for any $X \in \fg^0$. 
Thus $\exp(\fg^0)$ acts on the set $\MC(\fg)$.

To make this situation well-defined, we need to make some extra assumptions about the Lie algebra $\fg^0$. 

\begin{dfn}
Let $\fh$ be an ordinary Lie algebra.
We say $\fh$ is {\bf nilpotent} if its lower central series terminates at some finite order. 
That is, there exists $N \in \ZZ_{\geq 0}$ such that 
\[
[X_1,[X_2,[\cdots[X_N,Y]\cdots]]] = 0
\]
for all $X_1,\ldots,X_N, Y \in \fh$.
\end{dfn}

If $\fh$ is nilpotent then for any $X, Y \in \fh$ the Baker-Campell-Hausdorff formula for $\exp(X)\exp(Y)$ converges. 
In fact, one has the following.

\begin{prop}
If $\fh$ is a nilpotent Lie algebra with corresponding simply connected Lie group $H$, then the exponential map
\[
\exp : \fh \to H
\] 
is a diffeomorphism.
\end{prop}

We can now make the above arguments precise. 
We say a dg Lie algebra $\fg$ is nilpotent if its degree zero sub Lie algebra $\fg^0 \subset \fg$ is nilpotent. 
Let $\dgLienil$ be the category of nilpotent dg Lie algebras.
Recall that every graded piece $\fg^n$ of $\fg$ is a module for the Lie algebra $\fg^0$. 

\begin{lem}
Suppose $\fg$ is a dg Lie algebra such that $\fg^0$ is nilpotent.
Let $G^0$ be the corresponding simply connected Lie group. 
Then, the action of $\fg^0$ on $\fg^n$ exponentiates to an action of $G^0$ on $\fg^n$ for each $n$. 
Moreover, $G^0$ preserves the solutions to the Maurer-Cartan equation in $\fg^1$.
\end{lem}

There is an explicit formula for this action given by the BCH formula. 
For an original reference see the paper \cite{GM}. 
Given $X \in \fg^0$ and $\alpha \in \fg$, we have
\[
\exp(X) \cdot \alpha = \alpha + \sum_{n \geq 0} \frac{{\rm ad}(X)^n}{(n+1)!} ([X, \alpha] - \d X) .
\] 
Evidently, the right hand side only makes sense if $\ad(X)^n = 0$ for $n$ large enough.
This is where nilpotence is essential!
For a nice overview of nilpotence for dg Lie algebras see \cite{Getzler}. 

\begin{rmk}
Note that if $\fg^0$ is nilpotent, then the action of $\exp(X)$ corresponding to an element in $X \in \fg^0$ on $\fg^1$ makes sense even if $\fg^0$ is infinite dimensional. 
\end{rmk}

\begin{lem}
Suppose $\alpha \in \MC(\fg)$ and consider the dg Lie algebra $\fg_\alpha$ as in Lemma \ref{lem: easy}. 
If $\fg^0$ is nilpotent, then for any $X \in \fg^0$ there is an isomorphism of dg Lie algebras
\[
\varphi_X : \fg_{\alpha} \xto{\cong} \fg_{\exp(X) \cdot \alpha}
\]
defined by $\varphi_X(y) = \exp(X) \cdot y$. 
\end{lem}

\bibliographystyle{alpha}
%\bibliographystyle{spmpsci}  
\bibliography{def}

\end{document}
