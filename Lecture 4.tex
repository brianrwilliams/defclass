\documentclass[11pt]{amsart}

\usepackage{macros}

\linespread{1.25}

\usepackage[final]{pdfpages}

\setcounter{tocdepth}{2}

\title{Lecture 4: More on dg Lie algebras}

\begin{document}

\section{Deformation problems from dg Lie algebras}

We are ready to connect dg Lie algebras to the version of deformation theory that we met in the first few lectures. 

\subsection{From dg Lie to formal moduli}

\begin{dfn}
A {\bf commutative dg algebra (cdga)} is a dg vector space $(V,\d)$, together with a product $V \times V \to V \;,\; (x,y) \mapsto xy$ of degree zero, such that 
\begin{enumerate}
\item[(1)] Graded commutative. 
For any $x,y \in V$ one has $xy = (-1)^{|x||y|} y x$;
\item[(2)] Derivation. $\d (xy) = (\d x) y + (-1)^{|x|} x \d y$.
\end{enumerate}
A morphism of cdgas is a linear map preserving the differential and product. 
Denote the category of cdgas by $\dgCAlg_k$.
\end{dfn}

\begin{dfn} Suppose $(R,\d_R)$ is a cdga and $(\fg,\d_{\fg}, [-,-]_\fg)$ is a dg Lie algebra. 
Define the dg Lie algebra $(\fg \tensor_k R, \d, [-,-])$ where the differential is 
\[
\d = \d_\fg \tensor 1 + 1 \tensor \d_{R}
\]
and the bracket is defined by
\[
[x \tensor a, y \tensor b] = [x,y]_\fg \tensor ab .
\] 
This construction defines a functor
\[
\begin{array}{cccc}
\fg \tensor (-) : & \dgCAlg_k & \to & \dgLie_k \\
& R & \mapsto & \fg \tensor R .
\end{array}
\]
for any dg Lie algebra $\fg$. 
\end{dfn}

\begin{rmk}
Given any two dg vector spaces $V, W$ we can define there tensor product $V \tensor_k W \in \dgVect_k$. 
Its graded pieces are given by
\[
(V \tensor_k W)^m = \{V^i \tensor_k W^j\}_{i+j = m}
\]
where $V^i \tensor_k W^j$ is the ordinary tensor product of vector spaces.
The differential is
\[
\d_{V \tensor W} (v \tensor w) = \d_V(v) \tensor w + (-1)^{|v|} v \tensor \d_W (w) .
\]
In the definition above, we are simply taking the tensor product of underlying dg vector spaces. 
\end{rmk}

The following lemma is an easy exercise. 

\begin{lem}
Let $\fg$ be a dg Lie algebra and suppose $(A,\fm_A)$ is an Artinian algebra.
Consider the dg Lie algebra $\fg \tensor A$. 
Then, the sub dg Lie algebra
\[
\fg \tensor \fm_A \subset \fg \tensor A
\]
is nilpotent.
\end{lem}

This lemma implies that when restricted to Artinian algebras, the functor $\fg \tensor (-) : \CAlg_k \to \dgLie_k$ defines a functor
\[
\begin{array}{cccc}
\fg \tensor (-) : & \Art_k & \to & \dgLienil_k \\
& (A,\fm) & \mapsto & \fg \tensor \fm_A .
\end{array}
\]

This allows us to enhance the construction of the Maurer-Cartan elements of a dg Lie algebra $\MC(\fg)$ to a functor.

\begin{dfn}
Let $\fg$ be any dg Lie algebra.
Define the {\bf Maurer-Cartan} functor $\MC_\fg$ to be the composition
\[
\begin{array}{cccccccc}
\MC_\fg: & \Art_k & \xto{\fg \tensor (-)} & \dgLienil_k & \xto{\MC(-)} & \Set \\
& (A,\fm_A) & \mapsto & \fg \tensor \fm_A & & \\
& & & \fh & \mapsto & \MC(\fh) .
\end{array}
\]
Explicitly, this composition can be read off as $A \mapsto \MC(\fg \tensor \fm_A)$. 
\end{dfn}

The functor $\MC_\fg$ enjoys a lot of nice properties. 
First of all, it is functorial in the sense that every map of dg Lie algebras $\fg \to \fh$ induces a natural transformation $\MC_\fg \to \MC_\fh$. 
Also, it is an immediate observation that $\MC_\fg(k) = \{\star\}$ is the set with one element. 
Next, it preserves a well-behaved class of pull-backs. 

\begin{ex}
\label{ex: MC}
Let $\sigma : B \to A, \sigma' : C \to A$ be morphisms of Artinian algebras where $\sigma$ is surjective.
Then, the natural map
\[
\MC_\fg(B \times_A C) \xto{\cong} \MC_\fg(B) \times_{\MC_\fg(A)} \MC_\fg(C)
\]
is an isomorphism.
\end{ex}
%\begin{proof}
%Injectivity is obvious.
%Suppose that $\beta \times \gamma \in \MC_\fg(B) \times_{\MC_\fg(A)} \MC_\fg(C)$.
%Unravelling this, we have an element $\beta \in \MC(\fg \tensor \fm_B)$ and $\gamma \in \MC(\fg \tensor \fm_C)$ which satisfy $\sigma(\beta) = \sigma'(\gamma)$. 
%Since $\sigma
%\end{proof}|

Similarly, taking into account gauge transformations, we can make the following definition.
To state it, recall that when $\fg$ is a nilpotent Lie algebra that there is a group action by elements $X \in \fg^0$ through their exponential $\exp(X)$ (which is well-defined by nilpotency). 
This group action preserves the set of Maurer-Cartan elements. 

\begin{dfn}
Let $\fg$ be any dg Lie algebra.
Define the associated {\bf deformation} functor
\[
\begin{array}{cccc}
\Def_\fg : & \Art_k & \to & \Set \\
& (A,\fm_A) & \mapsto & \frac{\MC(\fg \tensor \fm_A)}{\exp((\fg \tensor \fm_A)^0)} .
\end{array}
\]
This functor assigns to every Artinian algebra $A$ the quotient of the set of Maurer-Cartan elements of $\fg \tensor \fm_A$ by the equivalence relation induce by gauge equivalence from elements in $(\fg \tensor \fm_A)^0 = \fg^0 \tensor \fm_A$. 
\end{dfn}

Note that the deformation functor is functorial in the sense that every map of dg Lie algebra $\fg \to \fh$ induces a natural transformation $\Def_\fg \to \Def_\fh$. 
Note that there is an obvious natural transformation of functors
\[
[-] : \MC_\fg \to \Def_\fg .
\] 


\begin{prop}
The pre-deformation functor $\Def_\fg$ is a deformation functor.
\end{prop}

\begin{proof}
We need to check that if $\sigma : B \to A$, $\sigma' : C \to A$ are maps of Artinian algebras with $\sigma$ surjective that the induced map
\beqn\label{defmap}
\Def_\fg (B \times_A C) \to \Def_\fg(B) \times_{\Def_\fg(A)} \Def_\fg(C)
\eeqn
is:
\begin{itemize}
\item[(A)] an isomorphism if $A = k$;
\item[(B)] surjective otherwise. 
\end{itemize} 
The first case (A) is obvious since $\Def_\fg(B \times C) = \Def_{\fg}(B) \times \Def_\fg(C)$. 

In general, suppose $[\beta] \times [\gamma] \in \Def_\fg(B) \times_{\Def_\fg(A)} \Def_\fg(C)$.
By assumption, $\sigma([\beta]) = \sigma'([\gamma])$ in $\Def_\fg(A)$.\footnote{If $\alpha \in \MC(\fg)$ we denote by $[\alpha]$ its class in $\Def(\fg)$.}
Choose lifts $\Tilde{\beta} \in \MC_\fg (B)$, $\Tilde{\gamma} \in \MC_\fg(C)$ of the classes $[\beta], [\gamma]$ respectively.
There then exists $a \in \fg^0 \tensor \fm_A$ such that 
\[
\sigma'(\Tilde{\gamma}) = \exp(a) \cdot \sigma(\Tilde{\beta}) .
\] 
By surjectivity of $\sigma$, there exists $b \in \fg^0 \tensor \fm_B$ such that $\sigma(b) = a$. 

By construction, the element $\exp(b) \cdot \Tilde{\beta}$ satisfies $\sigma'(\Tilde{\gamma}) = \sigma(\exp(b) \cdot \Tilde{\beta})$ in $\MC_\fg(A)$. 
Thus, $(\exp(b) \cdot \Tilde{\beta}) \times \Tilde{\gamma}$ determines an element in $\MC_\fg(B) \times_{\MC_\fg(A)} \MC_\fg(C)$.
By Exercise \ref{ex: MC}, there exists a unique $\alpha \in \MC_\fg (B \times_A C)$ that lifts $(\exp(b) \cdot \Tilde{\beta}) \times \Tilde{\gamma}$. 
The element $[\alpha]$ maps to $\beta \times \gamma$ under the natural map (\ref{defmap}).
This shows axiom (B), so we are done. 
\end{proof}

Recall the category of formal moduli problems $\Moduli_k$. 
Its objects are formal moduli problems over $k$, that is, functors
\[
F : \Art_k \to \Set
\]
and the morphisms are given by natural transformations. 
It is an immediate consequence of our discussion above that we have constructed a functor
\[
\begin{array}{cccc}
\Def : & \dgLie_k & \to & \Moduli_k \\
& \fg & \mapsto & \Def_\fg .
\end{array}
\] 

\subsection{Tangent and obstruction spaces}

\begin{prop}
There are natural isomorphisms of tangent spaces
\[
T_{\MC_\fg} \cong Z^1(\fg)
\] 
and 
\[
T_{\Def_\fg} \cong H^1(\fg) .
\] 
\end{prop}
\begin{proof}
Recall, the tangent space of a formal moduli problem $F$ is simply its value on the ring of dual numbers $F(k[\epsilon]/\epsilon^2)$. 

For the first claim, we compute
\begin{align*}
\MC_\fg(k[\epsilon]/\epsilon^2) & = \MC(\fg \tensor \epsilon k) \\ & = \left\{\epsilon x \in \epsilon \fg^1 \; | \; \d (\epsilon x) + \frac{1}{2} [\epsilon x, \epsilon x] = 0 \right\} \\ & = \{x \in \fg^1 \; | \; \d x = 0\} ,
\end{align*}
since $\epsilon^2 = 0$. 

For the second claim, we recall from last lecture that the action of $X \in \fg^0$ on $\alpha \in \MC(\fg)$ (when it exists) is defined by 
\[
\exp(X) \cdot \alpha = \alpha + \sum_{n \geq 0} \frac{{\rm ad}(X)^n}{(n+1)!} ([X, \alpha] - \d X) .
\]
For the tangent space, we are looking at the quotient 
\[
T_{\Def_\fg} = \MC(\epsilon \fg) / \{\exp(\epsilon X) \; | \; X \in \fg^0\}
\] 
Since $\epsilon^2 = 0$ the only term that survives is $\exp(\epsilon X) \cdot (\epsilon \alpha) = \alpha - \d X$.
Thus, we see that $T_{\Def_\fg} = T_{\MC_\fg} / \d \fg^0 = Z^1 \fg / B^1 \fg = H^1(\fg)$. 
\end{proof}

\subsection{Aside on obstruction space}

Suppose we have a small exact sequence of Artinian algebras
\beqn\label{smallexact}
0 \to M \to B \xto{\sigma} A \to 0 .
\eeqn
We want to construct a map
\[
\theta_B : \MC_\fg(A) \to H^2(\fg) \tensor M .
\]
Given $x \in \MC_{\fg}(A) = \MC(\fg \tensor \fm_A)$, let $\Tilde{x} \in \MC_\fg(B) = \MC(\fg \tensor \fm_B)$ be a lift and consider the element 
\[
\theta'(x) = \d \Tilde{x} + \frac{1}{2}[\Tilde{x}, \Tilde{x}] \in \fg^2 \tensor \fm_B .
\] 
By assumption $\sigma(\theta'(x)) = 0$ and so $\theta'(x)$ determines an element $\theta(x) \in \fg^2 \tensor \fm_B$.
Moreover
\[
\d \theta(x) = [\d \Tilde{x}, \Tilde{x}] = [\theta(x), \Tilde{x}] - \frac{1}{2} [[\Tilde{x},\Tilde{x}],\Tilde{x}]  .
\]
The first term on the right-hand side is zero since $\fm_B \cdot M = 0$.
Moreover, the second term is zero by degree reasons. 
Thus, $\theta(x)$ is closed and determines an element $[\theta(x)] \in H^2(\fg) \tensor M$. 
It is an exercise to see that it does not depend on any choices. 

\begin{prop}
The assignment that sends any small exact sequence (\ref{smallexact}) to the map
\[
\theta_B : \MC_\fg(A) \to H^2(\fg) \tensor M
\]
defines a complete obstruction theory for $\MC_\fg$. 
\end{prop}

\subsection{Not the whole story}

We have just seen a nice explicit story relating the theory of dg Lie algebras to formal moduli problems. 
This correspondence, however, loses a lot of information.
For instance, the formal moduli problem $\MC_\fg$ depends only on the pieces of the dg Lie algebra in degrees $1, 2$.
The formal moduli problem $\Def_\fg$ depends only on degrees $0,1,2$ of the dg Lie algebra. 

A precise way of stating the failure of this map to be an equivalence will be given in Section \ref{sec: quasi} below.
It says that $\Def_\fg$ only depends on the cohomology, or {\em homotopy type}, of $\fg$.
The main idea is enhance the category of formal moduli problems to be sensitive to this sort of weak equivalence. 
This will lead us naturally to the world of $\infty$-categories and higher algebra. 
Before embarking on this, we will recount some more machinery for dg Lie algebras. 

\section{Formal aspects of dg Lie algebras}

\def\oblv{{\rm oblv}}

It is a good time to remind ourselves of the formal situation that we are in. 
We have introduced the categories of dg commutative, dg associative, and dg Lie algebras.
All were obtained by putting some extra structure on a dg vector space (or cochain complex).
Thus, there are forgetful functors
\[
\oblv_{C}, \oblv_{A}, \oblv_L : \dgCAlg_k, \dgAss_k, \dgLie_k \to \dgVect_k .
\] 

\subsection{A slew of functors}
\def\Tens{{\rm Tens}}

\subsubsection{The free algebras}

The free algebra functor is given by the tensor algebra 
\[
\begin{array}{cccc}
\Tens(-) : & \dgVect_k & \to & \dgAss_k \\
& V & \mapsto & \Tens(V) = \bigoplus_{n \geq 0} V^{\tensor n} . 
\end{array}
\]
One extends the differential by the rule
\[
\d(v_1 \tensor \cdots \tensor v_n) = \sum_{i = 1}^n (-1)^{|v_1|\cdots |v_{i-1}|} v_1 \tensor \cdots (\d v_{i}) \tensor \cdots \tensor v_n .
\]

\begin{fact}
The functor $\Tens(-)$ is {\em left adjoint} to the forgetful functor $\oblv_A : \dgAss_k \to \dgVect_k$. 
\end{fact}

The free commutative algebra is the symmetric algebra on the dg vector space. 
This is the functor
\[
\begin{array}{cccc}
\Sym(-) : & \dgVect_k & \to & \dgAss_k \\
& V & \mapsto & \Sym (V) = \bigoplus_{n \geq 0} \left(V^{\tensor n}\right)_{S_n} . 
\end{array}
\]
The differential is defined in the same way as the tensor algebra. 

\begin{fact}
The functor $\Sym(-)$ is {\em left adjoint} to the forgetful functor $\oblv_C : \dgCAlg_k \to \dgVect_k$.
\end{fact}

Finally, there is a free dg Lie algebra functor define as follows. 
Take the tensor algebra $\Tens(V)$ and look at the dg subspace $V = V^{\tensor 1} \subset \Tens(V)$. 
One defines ${\rm free}_{\rm Lie}(V)$ to be the sub dg Lie algebra of $\Tens(V)_L$ generated by this subspace $V$. 

\begin{fact} The functor ${\rm free}_{\rm L}$ is {\em left adjoint} to the forgetful functor $\oblv_L : \dgLie_k \to \dgVect_k$. 
\end{fact}
\subsubsection{The enveloping algebra}

The enveloping algebra functor is
\[
\begin{array}{cccc}
U : & \dgLie_k & \to & \dgAss_k \\
& \fg & \mapsto & \Tens(\fg) / \cJ
\end{array}
\]
where $\cJ$ is the dg ideal of $\Tens(\fg)$ generated by expressions of the form 
\[
x \tensor y - (-1)^{|x||y|} (y \tensor x) - [x,y].
\]

\begin{fact}
The enveloping algebra functor $U(-)$ is {\em left adjoint} to the forgetful functor 
\[
(-)_L : \dgAss_k \to \dgLie_k .
\]
\end{fact}

\begin{rmk}
Just like the tensor algebra, the enveloping algebra admits a natural increasing filtration
\[
F^{\leq k} U(\fg) = \left(\bigoplus_{0 \leq i \leq k} \fg^{\tensor i}\right) / \cJ .
\]
The associated graded ${\rm Gr} \; U(\fg)$ is naturally a dg algebra. 
In fact, it is a commutative dg algebra. 
By the Poincar\'{e}-Birkoff-Witt theorem, the natural linear map $\fg \to F^{\leq 1}U(\fg)$ induces an isomorphism of commutative dg algebras
\[
\Sym(\fg) = \bigoplus_{k \geq 0} \Sym^k(\fg) \xto{\cong} {\rm Gr} \; U(\fg) . 
\] 
\end{rmk}

\subsection{Quasi-isomorphisms}\label{sec: quasi}

A map of dg vector spaces (cochain complexes) $f : V \to W$ is a linear map of grading degree zero that commutes with the differentials. 
If $V = \{V^n\}_{n \in \ZZ}$, $W = \{W^n\}_{n \in \ZZ}$,
this is equivalent to the data of linear maps $f^n : V^n \to W^n$ for each $n$ such that $\d^n_W f^n = f^n \d^n_V$. 
Such a map induces a linear map in cohomology $H^*f : H^*(V) \to H^*(W)$. 

\begin{dfn}
A map of dg vector spaces $f : V \to W$ is a {\bf quasi-isomorphism} if $H^*(f) : H^*(V) \to H^*(W)$ is an isomorphism. 
In this case, we will write $f : V \xto{\simeq} W$. 
\end{dfn}

The definition for dg Lie algebras is similar. 
Note that a map of dg Lie algebra $f : \fg \to \fh$ is, in particular, compatible with the differentials on both sides. 
Thus, it induces a map in cohomology $H^*(f) : H^*(\fg) \to H^*(\fh)$.
This is a map of {\em graded} Lie algebras (dg Lie algebras with zero differential). 

\begin{dfn}
Let $f : \fg \to \fh$ be a map of dg Lie algebras. 
We say $f$ is a {\bf quasi-isomorphism} if the underlying map of dg vector spaces is a quasi-isomorphism.
\end{dfn}

This definition simply ports the notion of quasi-isomorphism of dg vector spaces to dg Lie algebras using the forgetful functor $\oblv_L : \dgLie_k \to \dgVect_k$. 
We can do the same trick for commutative and associative dg algebras.
Namely, a map of commutative/associative dg algebras is a quasi-isomorphism if and only if it is a quasi-isomorphism of underlying dg vector spaces. 

Given this, the following lemma is easy. 

\begin{lem}
A map of dg Lie algebras $f : \fg \to \fh$ is a quasi-isomorphism if and only if the induced map
\[
U(f) : U(\fg) \to U(\fh)
\]
is a quasi-isomorphism of associative dg algebras. 
\end{lem}
\begin{proof}
We consider the spectral sequences induced by the natural filtrations on the enveloping algebras of $\fg$ and $\fh$. 
The map $U(f)$ clearly preserves the natural filtration on the enveloping algebra and so it induces a map of spectral sequences. 
In particular, it induces a map at the $E_1$ page, which is simply the associated graded algebras
\[
{\rm Gr} \; U(f) : \Sym^*(\fg) \to \Sym^*(\fh) .
\] 
The map of spectral sequence converges to the map $U(f) : U(\fg) \to U(\fh)$. 

Now, it is an easy exercise to show that $\fg \to \fh$ is a quasi-isomorphism if and only if it induces a quasi-isomorphism of free commutative dg algebras $\Sym^*(\fg) \to \Sym^*(\fh)$. 
Thus, in one direction, if $\fg \to \fh$ is a quasi-isomorphism, we obtain a quasi-isomorphism at the $E_1$-page, and the result follows. 

In the other direction, if $U(\fg) \to U(\fh)$ is a quasi-isomorphism, then taking associated gradeds we obtain a quasi-isomorphism $\Sym^*(\fg) \to \Sym^*(\fh)$ and hence $\fg \to \fh$ is a quasi-isomorphism by the above remark. 
\end{proof}

\begin{thm}
Suppose $f : \fg \xto{\simeq} \fh$ is a quasi-isomorphism of dg Lie algebras. 
Then, the corresponding map of deformation functors
\[
\Def(f) : \Def_\fg \xto{\cong} \Def_\fh
\]
is an {\bf isomorphism}. 
\end{thm}


\section{Modules and (co)homology}

\subsection{(dg) Modules}

Recall that to every dg vector space $V$ we can associated an associative dg algebra $\End_k(V)$ of endomorphisms. 

\begin{dfn}
Let $\fg$ be a dg Lie algebra.
A {\bf dg $\fg$-module} is a dg vector space $M$ together with a map of dg Lie algebras
\[
\rho : \fg \to \End_k(M) .
\]
A map of dg modules is defined in the obvious way. 
Let $\dgMod_{\fg}$ denote the category of dg $\fg$-modules. 
\end{dfn}

\begin{rmk}
By the universal property of the enveloping algebra, we see that a dg module for $\fg$ is equivalent to a (left) dg module for the associative dg algebra $U(\fg)$. 
\end{rmk}

The category of modules is an abelian category in the obvious way. 
Also, there is the notion of tensor product. 

\begin{dfn}
Suppose $M,N$ are two dg $\fg$-modules.
Define the tensor product dg $\fg$-module $M \tensor N$ to be the tensor product of underlying dg vector spaces with $\fg$-module structure given by
\[
\rho_{M} \tensor 1 + 1 \tensor \rho_N : \fg \to \End_k(M \tensor_k N) = \End_k(M) \tensor_k \End_k(N) .
\]
\end{dfn}

\begin{rmk}
If we think of $M,N$ as dg $U\fg$-modules, then 
\[
M \tensor N = M \tensor_{U\fg} N
\]
as $U \fg$-modules. 
\end{rmk}

\subsection{(Co)Homology}

As for ordinary modules, we have a pair of functors
\[
\begin{array}{cccl}
(-)^\fg : & \dgMod_\fg & \to & \dgVect_k  \\
& M & \mapsto & M^\fg = \{m \in M \; | \; x \cdot m = m \; , \; \forall x \in \fg\} \\
(-)_{\fg} : & \dgMod_{\fg} & \to & \dgVect_k \\
& M & \mapsto & M / \fg \cdot M 
\end{array}
\]
called the invariants/coinvariants respectively. 

\begin{rmk}
Note that 
\[
M^\fg = {\rm Hom}_{U \fg}(k, M)
\]
and 
\[
M_\fg = k \tensor_{U \fg} M .
\] 
\end{rmk}

\begin{lem}
Consider the functor
\[
{\rm triv}_\fg : \dgVect_k \to \dgMod_\fg
\]
that sends a dg vector space to the trivial $\fg$-module.
The functor $M \mapsto M^\fg$ is left adjoint to ${\rm triv}_\fg$. 
The functor $M \mapsto M_\fg$ is right adjoint to ${\rm triv}_\fg$. 
\end{lem}

As a consequence, taking invariants/coinvariants is left/right exact respectively. 
This motivates the following definition. 

\begin{dfn}
Let $\fg$ be a dg Lie algebra.
The {\bf Lie algebra homology} of $\fg$ is the left derived functor of coinvariants
\[
\begin{array}{cccc}
H_*(\fg ; -) : & \dgMod_\fg & \to & \Vect_k \\ 
& M & \mapsto & \LL_*(-)_\fg (M) \\ .
\end{array}
\]
Similarly, the {\bf Lie algebra cohomology} of $\fg$ is the right derived functor of invariants
\[
\begin{array}{cccc}
H^*(\fg ; -) : & \dgMod_\fg & \to & \Vect_k \\ 
& M & \mapsto & \RR_*(-)^\fg (M) \\ .
\end{array}
\]
\end{dfn}

\begin{rmk}
Using the Tor and Ext notation, we can write
\[
H^{\rm Lie}_*(\fg ; M) = {\rm Tor}^{U \fg}_* (k, M)
\]
and 
\[
H_{\rm Lie}^*(\fg ; M) = {\rm Ext}^*_{U \fg}(k, M) .
\] 
\end{rmk}

To compute the Lie algebra homology, for instance, one uses the usual trick for derived functors. 
By first finding a projective resolution, tensoring, then computing the cohomology. 
We proceed by finding a projective resolution of $U\fg$ 

\subsubsection{}

First, we sketch the following general construction for dg Lie algebras.
Given a dg Lie algebra $\fg$ define its {\em cone} to be the dg Lie algebra ${\rm Cone}(\fg)$ to be
\[
{\rm Cone}(\fg)_n = \fg_n \oplus \fg_{n-1}
\]
with differential
\[
\d_n = \left(\begin{array}{cc} \d_{\fg,n} & {\rm id}_{\fg_{n}} \\ 0 & \d_{\fg, n-1} \end{array}\right) : {\rm Cone}(\fg)_{n} = \fg_n \oplus \fg_{n-1} \to \fg_{n+1} \oplus \fg_n = {\rm Cone}(\fg)_{n+1}
\]
and bracket
\[
[(x,y), (x',y')] = \left([x,y]_{\fg}, [x,y'] + [y,x']\right) .
\]

\begin{lem}
There is a natural map of dg Lie algebras
\[
\fg \hookrightarrow {\rm Cone}(\fg) .
\]
Furthermore, ${\rm Cone}(\fg)$ is acyclic.
\end{lem} 

By functoriality of the enveloping functor, the associative algebra $U({\rm Cone}(\fg))$ is acyclic and hence a resolution for the trivial $U \fg$-module.
Thus, we have
\[
H^{\rm Lie}_*(\fg, M) = H^*\left(U({\rm Cone}(\fg)) \tensor_{U \fg} M \right) .
\] 

\begin{dfn}
The Chevalley-Eilenberg complex computing Lie algebra {\em homology} is the dg vector space
\[
\clieu_*(\fg ; M) := U({\rm Cone}(\fg)) \tensor_{U \fg} M .
\] 
Its cohomology is precisely the Lie algebra homology of $M$.
\end{dfn}

\begin{rmk}
Explicitly, as a graded vector space, the CE complex is of the form
\begin{align*}
\clieu_*(\fg ; M) & = U({\rm Cone}(\fg)) \tensor_{U \fg} M  \\ & = \left(\Sym(\fg[1]) \tensor_k U(\fg)\right) \tensor_{U \fg} M \\ &= \Sym(\fg[1]) \tensor_k M .
\end{align*}
Tracing through these isomorphisms, one can deduce that the differential is
\begin{align*}
\d_{CE} (x_1,\ldots, x_n) & = \sum_{i = 1}^n (\pm) x_1 \cdots x_{i-1} (\d x_i) x_{i+1} \cdots x_n \\ & + \sum_{i < j} (\pm) x_1\cdots \Hat{x_i} \cdots x_{j-1} [x_i,x_j] x_{j+1} \cdots x_n .
\end{align*}
\end{rmk}

\subsubsection{}

There is a completely analogous construction for Lie algebra cohomology. 

\begin{dfn}
The Chevalley-Eilenberg complex computing Lie algebra {\em cohomology} is the dg vector space
\[
\clie^*(\fg ; M) := U({\rm Cone}(\fg)) \tensor_{U \fg} M .
\] 
Its cohomology is precisely the Lie algebra homology of $M$
\[
H_{\rm Lie}^*(\fg ; M) = H^*\left(\Hom_{U \fg}(U({\rm Cone}(\fg)), M)\right) .
\]
\end{dfn}

\begin{rmk}
One can identify $\clie^*(\fg ; M)$ with a complex of the form
\[
\clie^*(\fg ; M) = \left(\Sym(\fg^\vee[-1]) \tensor_k M , \d^{CE}\right) .
\]
Note that, when $M = k$ there is an identification 
\[
\clie^*(\fg ; k) = \left(\clieu_*(\fg ; k)\right)^\vee = \Hom_k(\clieu_*(\fg ;k), k) .
\] 
\end{rmk}

\subsubsection{}

The CE complexes for homology and cohomology are functorial in the module input.
They both determine functors
\[
\begin{array}{cccc}
\clieu_*(\fg ; -) : & \dgMod_\fg & \to & \dgVect_k\\
\clie^*(\fg ; -) : & \dgMod_{\fg} & \to & \dgVect_k .
\end{array}
\]

We are interested in a different type of functoriality in the case that $M = k$, the trivial module. 
In this case, we write $\clieu_*(\fg) = \clieu(\fg;k)$ and similarly for cohomology.
A silly statement is that this trivial module is {\em universal} in the sense that it is a module for all dg Lie algebras.
Thus, we can contemplate the functoriality of homology/cohomology in the Lie algebra factor. 

\begin{lem}
The CE complex for homology/cohomology determine functors
\[
\begin{array}{cccc}
\clieu_*(-) : & \dgLie_k & \to & \dgVect_k\\
\clie^*(-) : & \dgLie_k & \to & \left(\dgVect_k\right)^{op} .
\end{array}
\]
Moreover, if $f : \fg \to \fh$ is a quasi-isomorphism of dg Lie algebras, then the induced maps
\[
\begin{array}{cccc}
\clieu_*(f) : & \clieu_*(\fg) & \to & \clieu_*(\fh)\\
\clie^*(\fg) : & \clie^*(\fh) & \to & \clie^*(\fg) 
\end{array}
\]
are quasi-isomorphisms. 
\end{lem}

\end{document}


