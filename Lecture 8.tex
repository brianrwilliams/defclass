\documentclass[11pt]{amsart}

\usepackage{macros}

\linespread{1.25}

\usepackage[final]{pdfpages}

\setcounter{tocdepth}{2}

\def\Fun{{\sf Fun}}
\def\Top{{\sf Top}}
\def\colim{{\rm colim}\;}
\def\Sing{{\rm Sing}}
\def\Cat{{\sf Cat}}

\title{Lecture 8: Simplicial Sets, I}

\begin{document}
\maketitle
\section{Definitions}

Let $\Delta$ be the following category. 
\begin{itemize}
\item Objects: totally ordered sets
\[
[n] = \{0 < 1 < \cdots < n\}
\]
where $n \geq 0$ is any non-negative integer.

\item Morphisms: order preserving maps of sets $\theta : [n] \to [m]$. 
That is, $i < j \implies \theta(i) < \theta(j)$. 
\end{itemize}

\begin{rmk}
If we think about the objects of $\Delta$ as categories themselves
\[
[n] = \{0 \to 1 \to \cdots \to n\}
\]
then the morphisms in $\Delta$ are precisely the functors. 
\end{rmk}

We refer to $\Delta$ as the {\em finite ordinal category}. 

\begin{dfn}
A {\bf simplicial set} is a functor
\[
X : \Delta^{op} \to \Set .
\]
In general, if $\sC$ is any category, a {\em simplicial object in $\sC$} is a functor 
\[
X : \Delta^{op} \to \sC .
\]
\end{dfn}

If $X$ is a simplicial set, we let $X[n]$ denote the image of the object $[n] \in \Delta$. 
The category of simplicial sets is the functor category
\[
\Set_{\Delta} = \Fun(\Delta^{op}, \Set) .
\]
Unwinding the definition, a morphism of simplicial sets $f : X \to Y$ is equivalent to prescribing a map of sets
\[
f[n] : X[n] \to Y[n]
\]
for each $n$ satisfying some conditions that we spell out below.

\begin{eg}\label{eg: sing}
Singular set.
Let $\Top$ be the category of topological spaces, and let
\[
|\Delta^n| = \{(t_0,\ldots,t_n) \in (\RR_{\geq 0})^{n+1} \; | \; \sum_{i\geq 0} t_i = 1\}
\]
be the standard $n$-simplex. 
For any order preserving map $\theta : [n] \to [m]$ there is a natural map of spaces
\[
\theta_* : |\Delta^n| \to |\Delta^m| .
\]
This defines a functor $\Delta \to \Top, [n] \mapsto |\Delta^n|$. 
Further, for any topological space $Y \in \Top$, we have the simplicial set
\[
\begin{array}{cccl}
{\rm Sing} (Y) : & \Delta^{op} & \to & \Set \\
& [n] & \mapsto & \Hom_{\Top}(|\Delta^n|, Y)
\end{array}
\]
called the {\em singular simplicial set} associated to $Y$. 
\end{eg}

\begin{eg} 
Standard simplex.
For each $n$ let 
\[
\Delta^n = \Hom_{\Delta}(-, [n]) : \Delta^{op} \to \Set .
\]
This is called the {\em standard $n$-simplex}. 
Note that this is simply the image of $[n]$ under the Yoneda embedding
\[
\Delta \to \Fun(\Delta^{op}, \Set) = \Set_{\Delta} .
\]
In particular
\[
\Hom_{\Set_\Delta} (\Delta^n, X) = X[n]
\]
for each $n$. 
\end{eg}

In pictures, we can think about $\Delta^1$ as simply the two vertex diagram
\[
\Delta^1 \;\; : \;\; 0 \to 1
\]
and $\Delta^2$ as being prescribed by the following diagram
\[
\begin{tikzcd}
& & 1 \ar[dr] & \\
\Delta^2 \; :  & 0 \ar[ur] \ar[rr] && 2 . 
\end{tikzcd}
\]
For instance, 
\[
\Delta^2 ([1]) = \Delta([1],[2]) = \{0 \to 1, 1 \to 2, 1 \to 2\}
\]
reads off all the $1$-simplices inside $\Delta^2$. 

\subsection{A combinatorial description}

So far, the definition of a simplicial set might seem a little abstract. 
One can prescribe the data of a simplicial set more explicitly as a collection of sets
\[
\{X[n]\}_{n \geq 0}
\]
together with the following data of {\em coface} and {\em codegeneracy} maps. 




\section{Geometric Realization}
We have already constructed a functor ${\rm Sing} (-) : \Top \to \Set_{\Delta}$ given by the singular set. 
One utility of simplicial sets is that they give us combinatorial rules for gluing spaces together in terms of simplices. 
Geometric realization makes this precise. 

I'm going to write the abstract definition, then interpret it in a way that hopefully makes clear what is going on.

Given any category $\sC$ and object $X \in \sC$ one can consider the {\em overcategory} (or slice category) $\sC_{/X}$ whose objects are maps
\[
Y \to X 
\]
and whose morphisms are commuting triangles
\[
\begin{tikzcd}
Y \ar[rr] \ar[dr] & & Y' \ar[dl] \\
& X & 
\end{tikzcd}
\]
There is a dual notion of an undercategory. 

Given a simplicial set $X$ we can then consider the overcategory $\Set_{\Delta / X}$.
Think about this as the category of maps of simplical sets into $X$.
We want to think about a smaller category, namely the category of maps of just the {\em $n$-simplices} into $X$, that we denote $\Delta_{/ X}$. 
This category fits into a commutative diagram
\[
\begin{tikzcd}
\Delta_{/X} \ar[r] \ar[d] & \Set_{\Delta / X} \ar[d] \\
\Delta \ar[r, "\Delta^{(-)}"] & \Set_{\Delta}
\end{tikzcd}
\]
The lower horizontal map is the Yoneda embedding sending $[n] \mapsto \Delta^n$. 
The right vertical map is the map that forgets the map into $X$ defining the object of the overcategory and just remembers the source object. \footnote{This is a pullback diagram in the category of categories.}

Concretely, the objects $\Delta_{/X}$ consists of maps, for each $n$
\[
\Delta^n \to X .
\]
These are precisely the $n$-simplices $X[n]$ of $X$. 
The morphisms are commuting triangles of simplicial sets
\[
\begin{tikzcd}
\Delta^n \ar[rr] \ar[dr] & & \Delta^m \ar[dl] \\
& X & 
\end{tikzcd}
\]

Note that any simplicial set defines a composition of functors
\[
\Delta_{/X} \to \Delta \xto{\Delta^{(-)}} \Set_{\Delta} .
\]
The following lemma consists of throwing definitions around

\begin{lem}
There is an isomorphism of simplicial sets
\[
\colim \left(\Delta_{/X} \to \Delta \xto{\Delta^{(-)}} \Set_{\Delta} \right) \xto{\cong} X
\]
\end{lem}

Roughly, this lemma just says that $X$ is glued together from its $n$-simplices.
Using this perspective, the following definition is properly motivated. 

\begin{dfn}
Let $X$ be a simplicial set. 
Define the {\bf geometric realization} of $X$ to be the topological space
\[
|X| := \colim \left(\Delta_{/X} \to \Delta \xto{|\Delta^{(-)}|} \Top \right) .
\]
Here, $|\Delta^{(-)}|$ is the functor from Example \ref{eg: sing} which sends $[n]$ to the geometric $n$-simplex $|\Delta^n| \subset \RR^n$.
\end{dfn}

We have already seen how colimits can be used to glue together a space.

Almost tautologically, one has the following.

\begin{prop}
The functor $|-|$ is left adjoint to $\Sing(-)$:
\[
\begin{tikzcd}
\Set_{\Delta} \ar[r,bend left,"|-|",""{name=A, below}] & \Top \ar[l,bend left,"\Sing",""{name=B,above}] \ar[from=A, to=B, symbol=\dashv] .
\end{tikzcd}
\]
\end{prop}

\section{We have the nerve}

Not only do simplicial sets give us a nice combinatorial description of topological spaces, but they provide a great organizational tool for many categorical constructions. 

Let $\Cat$ be the category of categories. 
Here, we mean the category whose objects are categories, and whose morphisms are functors. 
Recall, the objects of the ordinal category $\Delta$ consisted of ordered sets $[n]$ which we can think of as categories 
\[
[n] = \{0 \to 1 \to \cdots \to n\} .
\]
Thus, we have a functor $\Delta \hookrightarrow \Cat$. 

Abstractly, the {\em nerve} is a functor $N : \Cat \to \Set_{\Delta}$ given by the left Kan extension of $\Delta \hookrightarrow \Cat$ along the Yoneda embedding $\Delta \to \Set_{\Delta}$. 
\[
\begin{tikzcd}
\Delta \ar[r] \ar[d] & \Set_{\Delta} \\
\Cat \ar[ur, dotted, "N(-)"'] & .
\end{tikzcd} 
\]
A more explicit definition is the following. 

\begin{dfn}
Let $\sC$ be a category.
Define the simplicial set $N \sC$ whose $n$-simplices are all $n$-composable chains
\[
X_0 \to X_1 \to \cdots \to X_n
\]
in $\sC$. 
In other words, $N \sC [n] = \Hom_{\Cat}([n], \sC)$. 
\end{dfn}

\end{document}