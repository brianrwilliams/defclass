\documentclass[11pt]{amsart}

\usepackage{macros}

\linespread{1.25}

\usepackage[final]{pdfpages}

\setcounter{tocdepth}{2}

\def\Fun{{\sf Fun}}
\def\Top{{\sf Top}}
\def\colim{{\rm colim}\;}
\def\Sing{{\rm Sing}}
\def\Cat{{\sf Cat}}
\def\Spec{{\rm Spec}}

\title{Lecture 15: Augmented algebras}

\begin{document}
\maketitle

In order to formulate deformation theory at the level of $\infty$-categories, we first must provide a substitute for the category of Artinian algebras. 
For ordinary deformation theory, we saw that Artinian algebras were the correct test objects needed to define a formal scheme. 
In this lecture, we define the correct enhancement of Artinian algebras in the  $\infty$-category of commutative dg algebras $\CAlg_k$. 

\section{Overcategories and joins}

The data of an Artinian algebra is a pair $(A, \fm)$ where $A$ is a commutative algebra and $\fm$ is a maximal ideal satisfying $A / \fm = k$. 
In particular, there is a canonical map $A \to A / \fm = k$ for every such algebra. 
Moreover, a map of Artinian algebras must necessarily preserve these canonical maps. 

An algebra $A$ together with a map $\epsilon : A \to k$ is called {\em augmented}. 
Geometrically, augmented algebras correspond to pointed affine schemes
\[
* = \Spec(k) \to \Spec(A) .
\]

The first notion which we transfer to the $\infty$-category setting is that of an augmented algebra. 
To do this, we need to step aside and define what is meant by an {\em overcategory}. 

\subsection{$\infty$-overcategories}

If $\sC$ is any category and $X \in \sC$ is an object, we know how to define the overcategory $\sC_{/X}$. 
Its objects are arrows $Y \to X$ in $\sC$ and morphisms are commuting triangles. 

\begin{eg}
Let $\dgCAlg_k$ be the ordinary category of dg commutative algebras. 
Then $(\dgCAlg_k)_{/ k}$ is the ordinary category of augmented dg commutative algebras. 
\end{eg}

There is a way of defining the overcategory $\sC_{/X}$ via a universal property, which makes use of the following categorical construction. 

\begin{dfn}
Let $\sC,\sD$ be two categories. 
The {\bf join} of $\sC$ with $\sD$ is the category $\sC \star \sD$ whose objects are 
\[
{\rm ob}(\sC \star \sD) = {\rm ob} (\sC) \sqcup {\rm ob} (\sD)
\]
and morphisms are:
\[
\Hom_{\sC \star \sD} (X, Y) =  \left\{  \begin{array}{ccccc} \Hom_{\sC}(X,Y) & {\rm if} & X, Y \in \sC \\ 
										       \Hom_{\sD}(X,Y) & {\rm if} & X,Y \in \sD \\ 
										       \{\star\} & {\rm if} & X \in \sC , Y \in \sD \\
										       \emptyset  & {\rm if} & X \in \sD , Y \in \sC . \end{array} \right.
\]
\end{dfn}

Roughly, this is like the ``disjoint union" of two categories where we throw in a unique morphism from every object of $\sC$ to every object of $\sD$. 
Note that there is a natural functor
\[
\sD \to \sC \star \sD .
\]

\begin{ex} 
Recall the definition of the {\em join} of two topological spaces $S,S'$
\[
S \star T = (S \times T \times [0,1]) / \sim
\]
where the equivalence relation is $(a,b, 0) \sim (a,b', 0)$ and $(a,b,1) \sim (a',b,1)$ for all $a,a' \in S$ and $b,b' \in T$. 
This equivalence relation simply collapses $T$ to a point at $t = 0$ and collapses $S$ to a point at $t = 1$. 
In particular, both $S, T$ map into $S \star T$.

Given any category $\sC$ we can consider the simplicial set $N\sC$ given by the nerve. 
Show that
\[
|N\sC \star N \sD| \simeq |N \sC| \star |N \sD|
\]
where $|N \sC|$ is the geometric realization of the simplicial set $N \sC$. 
\end{ex}

Joins allow us to make the following universal property for overcategories. 
As above, let $X \in \sC$ be some object in a category.
We can view this as the data of a functor
\[
X : [0] \to \sC
\]
where $[0]$ is the category with a single object $0$ and no non-identity morphisms. 

\begin{lem}
The category $\sC_{/X}$ satisfies the following universal property: for any category $\sD$ one has an equivalence of categories
\[
\Fun(\sD, \sC_{/X}) = \Fun_{X} (\sD \star [0], \sC)
\]
where $\Fun_{X} (\sD \star [0], \sC)$ is the full subcategory of functors $G : \sD \star [0] \to \sC$ for which the composition 
\[
[0] \to \sD \star [0] \xto{G} \sC
\]
is $X$. 
\end{lem}
\begin{proof}
This is almost tautological. 
Suppose we have a functor $G : \sD \star [0] \to \sC$ whose restriction to $[0]$ is the constant functor based on $X$.
Define a functor $G |_{\sD} : \sD \to \sC_{/X}$ as follows.
To any object $d \in \sD$ we have a unique map $d \to 0$ in $\sD \star [0]$ by definition. 
Then $G|_{\sD} (d) = G(d \to 0) = \left(G(d) \to G(0) = X\right) \in \sC_{/X}$. 
It is an exercise to see how this defines a functor. 
\end{proof}

\begin{ex} 
Modify this lemma to characterize the overcategory $\sC_{/F}$ where $F$ is any functor $F : \sC' \to \sC$ by a similar universal property.  
\end{ex}

\subsubsection{}

One can also define the join of two simplicial sets via the same formula as for topological spaces. 
In this way, joins commute with geometric realization in the sense: if $K, K'$ are two simplicial sets, then
\[
|K \star K'| \simeq |K| \star |K'| .
\]

\begin{lem}
If $\sC$ and $\sD$ are $\infty$-categories, then so is $\sC \star \sD$. 
\end{lem}

With this definition, we can define the over category of an $\infty$-category over an object in a completely analogous way as the usual case. 

\begin{dfn}
Let $\sC$ be an $\infty$-category and $X \in \sC$ an object. 
Define the $\infty$-category $\sC_{/X}$ by the universal property: for any $\infty$-category $\sD$, $\sC_{/X}$ is the $\infty$-category so that there is a natural equivalence
\[
\Fun(\sD, \sC_{/X}) \simeq \Fun_X(\sD \star \Delta^0, \sC)
\]
where the right-hand side consists of maps of simplicial sets $G : \sD \star \Delta^0 \to \sC$ such that the induced map
\[
\Delta^0 \to \sD \star [0] \xto{G} \sC
\]
is $X$. 
\end{dfn}

\begin{ex}
Note that an object in an $\infty$-category is nothing but a map of simplicial sets $X : \Delta^0 \to \sC$. 
Define the notion of an $\infty$-category over a functor $F : \sC' \to \sC$ where $\sC'$ is any other $\infty$-category. 
\end{ex}

One can also define, in a completely analogous way, the notion of an {\em under} category in the $\infty$-categorical sense. 
The key idea is that instead of $\sD \star \Delta^0$, one uses $\Delta^0 \star \sD$ in the universal property. 

\section{Augmentations}

We can now define the $\infty$-categorical notion of an augmentation for a commutative algebra. 

\begin{dfn}
Define the $\infty$-category of {\bf augmented} commutative algebras by
\[
\CAlg_k^{\rm aug} = \left(\CAlg_k\right)_{/k} 
\]
where $\CAlg_k$ is the $\infty$-category of all commutative algebras. 
\end{dfn}

\begin{rmk}
Recall, $\CAlg_k$ was constructed from the model category of commutative dg algebras $\dgCAlg_k$. 
There is an ordinary category of {\em augmented} commutative dg algebras $\CAlg_k^{\sf dg,aug}$, which are simply commutative dg algebras $A$ together with a map of commutative dg algebras $A \to k$. 
The maps are those of commutative dg algebras preserving the augmentation. 

There is a natural model structure on $\CAlg_k^{\sf dg,aug}$. 
The Dwyer-Kan localization of this model category produces an $\infty$-category equivalent to the one $\CAlg^{\rm aug}_k$ we have just defined. 
\end{rmk}


\end{document}